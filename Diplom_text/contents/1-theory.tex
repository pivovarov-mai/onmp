% Название разделов -- все прописные
\section{ТЕОРЕТИЧЕСКОЕ ОБОСНОВАНИЕ РАЗРАБОТКИ БАЗЫ ДАННЫХ}

\subsection{Анализ предметной области}

Анализ предметной области является первым этапом в проектировании базы данных, в ходе которого выделяются основные объекты и их свойства, определяются первоначальные требования и границы проекта, чтобы разработать эффективную и безопасную базу данных.

Предметной областью разрабатываемой модели данных является система заполнения электронных медицинских карт для отделения неотложной медицинской помощи.

Скорая медицинская помощь (СМП) – вид медицинской помощи, оказываемой гражданам при заболеваниях, несчастных случаях, травмах, отравлениях и других состояниях, требующих срочного медицинского вмешательства.

Отделение неотложной (экстренной) медицинской помощи больницы является структурным подразделением многопрофильной больницы, которое в круглосуточном режиме оказывает экстренную (неотложную) медицинскую помощь.

Медицинская помощь включает в себя процедуру проведения осмотра, непосредственно манипуляции по оказанию медицинской помощи, а также консультирование пациента по с целью определения наиболее эффективного, безопасного и экономически оправданного курса лечения.

ОНМП осуществляет следующие функции:

\begin{itemize}
    \item прием пациентов с острыми заболеваниями, несчастными случаями, травмами, отравлениями и другими состояниями, требующими немедленной медицинской помощи,
    \item проведение первичной медицинской диагностики и оценки состояния пациента, осуществление мер, направленных на стабилизацию состояния пациента,
    \item предоставление неотложной медицинской помощи, включая проведение лечебных манипуляций, инъекций, переливание крови, а также оказание психологической помощи,
    \item организация и координация работы других специалистов и служб в медицинском учреждении, если необходимо,
    \item подготовка пациента к транспортировке в стационар для дальнейшего лечения,
    \item соблюдение всех медицинских стандартов и требований, направленных на обеспечение безопасности пациентов и медицинского персонала,
    \item организация транспортировки пациентов в случае необходимости,
    \item обеспечение работы необходимого медицинского оборудования, инструментов, материалов и медикаментов, необходимых для оказания неотложной медицинской помощи,
    \item проведение медицинской документации, включая учет медицинских случаев, регистрацию историй болезни и медицинских записей,
    \item обеспечение мониторинга и контроля за состоянием пациентов, находящихся на лечении в отделении неотложной (экстренной) медицинской помощи,
    \item обучение медицинского персонала и сотрудников отделения неотложной медицинской помощи новым методам лечения и диагностики,
    \item сотрудничество и консультации с другими медицинскими учреждениями, специалистами и службами для повышения качества оказываемой медицинской помощи.
\end{itemize}

Отделение неотложной (экстренной) медицинской помощи является ключевым звеном в системе оказания медицинской помощи населению и обеспечивает быстрое и эффективное лечение при острых заболеваниях и травмах.

Базы данных для неотложной медицинской помощи являются критически важным элементом в работе современных медицинских учреждений. Они позволяют эффективно организовывать, хранить и обрабатывать медицинскую информацию, ускоряя процессы принятия решений и повышая качество медицинской помощи.

Основным требованием к базе данных для неотложной медицинской помощи является ее способность оперативно и точно хранить медицинскую информацию о пациентах. Эта информация может включать данные о медицинской истории пациента, диагнозе, принятых мероприятиях, результатах обследований и лекарственном лечении.

Для эффективного управления медицинской информацией в базе данных неотложной медицинской помощи необходима специализированная система управления базами данных (СУБД). Существует множество СУБД, которые могут использоваться для этой цели, включая Oracle, MySQL, Microsoft SQL Server, PostgreSQL и др.

Также необходимо учитывать специфику работы медицинских учреждений, которые могут иметь различные требования к хранению и обработке медицинской информации. Например, больницы могут иметь разные отделения, каждое из которых может иметь свои особенности в обработке информации. Кроме того, необходимо учитывать возможность интеграции с другими системами, такими как системы управления ресурсами и планирования процессов.

Важно также отметить, что база данных для неотложной медицинской помощи должна соответствовать требованиям законодательства в области защиты персональных данных, таких как GDPR в Европейском союзе и HIPAA в США. Регулярное обновление и адаптация базы данных к изменениям в законодательстве помогут поддерживать соответствие нормам и предотвращать правовые проблемы.

Исследования показывают, что эффективное использование баз данных в медицинских учреждениях может значительно улучшить качество медицинской помощи и снизить затраты на ее оказание. Например, использование баз данных может уменьшить количество ошибок в принятии решений, сократить время, необходимое для оказания медицинской помощи, и повысить точность диагностики.

На текущий момент существует множество различных подходов к проектированию баз данных для медицинских учреждений. Некоторые из них ориентированы на сохранение структурированной информации, другие на работу с полуструктурированными и неструктурированными данными.

Одним из наиболее распространенных подходов к проектированию баз данных для медицинских учреждений является использование реляционной модели данных. Реляционная модель данных позволяет хранить информацию в виде таблиц, которые могут быть связаны друг с другом через ключи. Это позволяет обрабатывать большие объемы структурированных данных и осуществлять запросы на выборку данных, используя язык SQL.

Тем не менее, в последние годы набирают популярность нереляционные базы данных, такие как MongoDB, Cassandra, Couchbase и др. Они хранят данные в более гибкой форме, позволяют более легко масштабировать базы данных и работать с полуструктурированными и неструктурированными данными, такими как тексты медицинских записей, изображения и видео.

Важным аспектом разработки баз данных для медицинских учреждений является также обеспечение защиты медицинской информации от несанкционированного доступа. Для этого применяются различные меры безопасности, такие как шифрование данных, многофакторная аутентификация и разграничение прав доступа к информации в зависимости от роли пользователя.

Наконец, важно отметить, что разработка баз данных для медицинских учреждений является сложным и многогранным процессом, требующим учета множества факторов, таких как специфика медицинских процедур, законодательство, требования к безопасности и другие. Поэтому необходимо проводить тщательный анализ предметной области и разрабатывать индивидуальные решения для каждого конкретного медицинского учреждения.



\subsection{Назначение и возможности базы данных}

В предметной области системы заполнения электронных медицинских карт для отделения неотложной медицинской помощи, основные объекты и свойства, которые следует рассмотреть, могут включать:

\begin{itemize}
    \item пациенты: информация о каждом пациенте, включая персональные данные (имя, дата рождения, пол и контактная информация), медицинскую историю, диагнозы, принятые мероприятия, результаты обследований и лекарственное лечение,
    \item медицинские работники: данные о врачах, медсестрах и других медицинских специалистах, включая их идентификационные данные, специализацию, график работы и доступные привилегии,
    \item медицинские процедуры: информация о проводимых процедурах, включая коды процедур, описания, стоимость, требуемое оборудование и прочие детали,
    \item отделения: данные о различных отделениях неотложной медицинской помощи в больнице, их назначение, доступный персонал и оборудование,
    \item ресурсы: информация о доступных ресурсах, таких как медицинское оборудование, лекарства, материалы и другие необходимые средства,
    \item расписание: график работы медицинского персонала и расписание доступности отделений и ресурсов,
    \item системы управления и интеграция: необходимо учесть возможность интеграции с другими системами, такими как системы управления ресурсами и планирования процессов, чтобы обеспечить эффективное взаимодействие и координацию деятельности медицинских учреждений.
\end{itemize}

Для лучшего понимания предметной области, рассмотрим конкретный пример - разработку базы данных для отделения неотложной медицинской помощи.

Медицинские работники оказывают медицинскую помощь пациентам с острыми заболеваниями и травмами, которые требуют немедленного вмешательства, также необходимо быстро и точно определить диагноз, назначить лечение и принять меры по сохранению жизни пациента.

Одной из основных задач базы данных для отделения неотложной медицинской помощи является хранение и обработка медицинских данных пациентов, включая информацию о симптомах, диагнозах, назначенных лекарствах, процедурах и т.д. Кроме того, необходимо учитывать, что пациенты могут обращаться за медицинской помощью в нескольких отделениях неотложной медицинской помощи, поэтому база данных должна позволять обмениваться информацией между различными медицинскими учреждениями.

Для решения этих задач можно использовать реляционную базу данных, в которой каждый пациент будет представлен в виде отдельной записи в таблице, содержащей данные о пациенте, диагнозах, лекарствах и т.д. Ключами в таблицах могут быть номера пациента, номера записи и т.д. Это позволит производить выборку данных о конкретном пациенте, обращаться к истории его болезни, проводить анализ данных и выявлять тенденции в заболеваемости и лечении.

Однако, реляционная модель может столкнуться с проблемами при работе с полуструктурированными и неструктурированными данными, такими как медицинские изображения, видео и тексты медицинских записей. Для работы с этими данными может использоваться нереляционная база данных, такая как MongoDB. В MongoDB данные могут быть храниться в более гибкой форме, используя форматы, такие как JSON и BSON. MongoDB также позволяет хранить и обрабатывать файлы в формате BLOB (binary large object), что делает его идеальным инструментом для хранения медицинских изображений и других неструктурированных данных.

Еще одной важной задачей при разработке базы данных для отделения неотложной медицинской помощи является обеспечение безопасности хранения и доступа к медицинским данным. Для этого может использоваться различные меры, такие как шифрование данных, авторизация пользователей и аудит доступа.

Также при проектировании базы данных для отделения неотложной медицинской помощи необходимо учитывать требования к ее масштабируемости, отказоустойчивости и производительности. В случае большого количества пациентов и медицинских записей может потребоваться использование кластерной архитектуры, репликации данных и других технологий, позволяющих обеспечить высокую доступность и производительность базы данных. Также необходимо обрабатывать большие объемы данных и поддерживать быстрый доступ к ним.

На сегодняшний день существует множество различных систем управления базами данных, которые могут быть использованы для разработки базы данных для отделения неотложной медицинской помощи, включая MySQL, PostgreSQL, Oracle Database, Microsoft SQL Server, MongoDB и др. Выбор конкретной системы управления базами данных зависит от требований к производительности, масштабируемости, отказоустойчивости и других факторов.

В заключение, разработка базы данных для отделения неотложной медицинской помощи является сложной задачей, которая требует учета множества факторов, таких как безопасность, масштабируемость, отказоустойчивость и производительность. Решение этих задач может потребовать использования различных технологий и систем управления базами данных, а также тщательного анализа требований и потребностей пользователей.

Анализируя все вышеперечисленные требования к организации базы данных, можно сделать вывод, что основным субъектом данной базы данных является врач, который будет осуществлять выезд и непосредственное оказание всех необходимых медицинских услуг.

Врач может иметь доступ к данным о препаратах, находящихся в использовании у данного наряда, он может осуществить поиск по названию лекарственного средства для оперативного предоставления ответа по запросу клиента.

Основные реализуемые функции:
\begin{itemize}
    \item аутентификация и авторизация,
    \item разграничение ролей пользователей,
    \item добавление и удаление данных,
    \item поиск данных по нескольким критериям,
    \item вывод вспомогательных информационных таблиц.
\end{itemize}



\textbf{Выводы по разделу}

В конечном итоге, проектирование базы данных для системы заполнения электронных медицинских карт является сложным процессом, требующим тщательного анализа предметной области, определения требований и границ проекта, выбора подходящей модели данных и СУБД, обеспечения безопасности данных, обучения пользователей и поддержки системы. Это комплексный процесс, который должен выполняться с участием экспертов в области медицины и баз данных для достижения оптимального результата. Результатом успешной разработки будет эффективная система управления медицинской информацией, способствующая повышению качества медицинской помощи, оптимизации процессов и улучшению результатов лечения пациентов.