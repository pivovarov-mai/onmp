\section{ЗАЩИТА БАЗЫ ДАННЫХ ОТ НЕСАНКЦИОНИРОВАННОГО ДОСТУПА}

\subsection{Конфиденциальность персональных данных}

При разработке базы данных важно учитывать требования безопасности и конфиденциальности медицинской информации. Для этого могут быть приняты следующие меры:

\begin{itemize}
    \item шифрование данных: чувствительная медицинская информация, такая как персональные данные пациентов и медицинская история, должна быть зашифрована при хранении и передаче. Использование сильных шифровальных алгоритмов поможет защитить данные от несанкционированного доступа,
    \item многофакторная аутентификация: для обеспечения безопасности доступа к базе данных, особенно для медицинского персонала, следует реализовать многофакторную аутентификацию. Это требует предоставления нескольких форм идентификации, таких как пароль, биометрические данные или специальные токены, для подтверждения легитимности пользователя,
    \item управление доступом: реализация гибкой системы управления доступом поможет ограничить доступ к конфиденциальной информации только уполномоченному персоналу. Ролевая модель доступа может быть использована для определения прав доступа на основе ролей и ответственностей сотрудников,
    \item аудит и мониторинг: важно вести аудит и мониторинг доступа к базе данных для обнаружения и предотвращения несанкционированного доступа или неправомерной активности. Журналы доступа и системы мониторинга позволят выявить подозрительную активность и принять соответствующие меры,
    \item cоответствие нормам и законодательству: разработка базы данных должна соответствовать применимым нормам и законодательству в области защиты персональных данных. Это включает в себя соблюдение правил по сбору, хранению, использованию и передаче медицинской информации,
    \item восстановление после катастрофы и резервное копирование: необходимо реализовать надежный план восстановления после катастрофы и резервного копирования, чтобы обеспечить доступность и целостность медицинской базы данных. Регулярные резервные копии должны выполняться, чтобы защитить от потери данных в случае сбоев системы, стихийных бедствий или кибератак. Также важно периодически тестировать процесс восстановления, чтобы убедиться в возможности точного и эффективного восстановления данных.
    
\end{itemize}

Анализируя вышеперечисленные пункты, можно сделать вывод, что обеспечение конфиденциальности хранения персональных данных является одним из ключевых вопросов, связанных с использованием цифровых медицинских карт и баз данных \cite{online6}..

Для обеспечения конфиденциальности данных необходимо проводить работу по защите информации и соблюдению законодательных требований. Организация должна предпринимать меры для защиты персональных данных пациентов, такие как использование паролей и шифрования данных. Кроме того, необходимо обучать медицинских работников основам информационной безопасности и контролировать доступ к информации.

Конфиденциальность хранения персональных данных пациентов - это важный аспект использования цифровых медицинских карт и организации должны принимать соответствующие меры для обеспечения безопасности информации и соблюдения законодательных требований.

PostgreSQL поддерживает множество функций для обеспечения безопасности данных, включая конфиденциальность хранения персональных данных.

Одним из ключевых механизмов для обеспечения конфиденциальности данных в PostgreSQL является использование различных методов шифрования, включая шифрование данных в пути и в покое, а также использование SSL для защиты соединений.

PostgreSQL также обеспечивает механизмы авторизации и аутентификации, которые позволяют контролировать доступ к базе данных и ее объектам, таким как таблицы и представления. Например, администраторы могут назначать различные роли и права доступа к объектам базы данных, что позволяет управлять доступом к конфиденциальным данным.

PostgreSQL также поддерживает аудиторскую функциональность, которая позволяет записывать действия пользователей в базе данных, такие как входы и выходы из системы, выполнение запросов и изменение данных. Это позволяет отслеживать и анализировать действия пользователей, чтобы обеспечить безопасность данных и защитить их от несанкционированного доступа.

В целом, PostgreSQL - это надежная и безопасная реляционная база данных, которая обеспечивает множество механизмов для защиты конфиденциальности данных, включая персональные данные пациентов.




\subsection{Использование регистрационных имен, имен пользователей, ролей и разрешений}

PostgreSQL обладает базовым механизмом защиты, включающим в себя идентификацию, аутентификацию и авторизацию. Идентификация – проверка на то, существует ли данный субъект, желающий воспользоваться данным ресурсом. Следующим шагом после идентификации следует аутентификация – проверка подлинности субъекта, в основном для этого используется паролевый метод. Конечным шагом является авторизация – предоставление необходимых прав субъекту или отказ в доступе к нужным ресурсам. Для реализации описанного механизма применяются регистрационные имена, которые инициализируются следующим образом: «CREATE ROLE login\_name WITH LOGIN PASSWORD 'login\_password' CREATEDB DEFAULT\_DATABASE db\_name;». Стоит остановиться на параметрах «CHECK\_EXPIRATION» и «CHECK\_POLICY». «CHECK\_EXPIRATION» отвечает за то, есть ли у пароля срок истечения, а «CHECK\_POLICY» определяет, должна ли использоваться паролевая политика. По умолчанию, проверка срока действия пароля и политики паролей включена в PostgreSQL, поэтому опции CHECK\_EXPIRATION и CHECK\_POLICY не нужно указывать.

После создания регистрационных имен для них требуется создать имена пользователей. Это требуется из-за того, что регистрационные имена нужны лишь для получения доступа на сервер, а для взаимодействия внутри сервера нужны уже вторые. Чтобы создать имя пользователя и привязать его к регистрационному имени можно применить следующий шаблон: «GRANT login\_name TO user\_name;».

Определив регистрационные имена и их пользователей, остается лишь предоставить им необходимые разрешения, позволяющие взаимодействовать с БД. Можно отдельно каждому пользователю прописывать разрешения, но в данном случае это нецелесообразно – потому что у всех медиков должны быть одинаковые права. Но стоит еще упомянуть такое понятие как роль. Роль – объект БД, позволяющий объединять пользователей в единую группу с целью эффективного администрирования. Для создания ролей используется следующий шаблон: «CREATE ROLE role\_name;». Для добавления пользователя в данную роль применяют следующий синтаксис: «GRANT role\_name TO user\_name;». Также в PostgreSQL есть возможность создавать ролевые комбинации на уровне базы данных, чтобы сделать связь логин-роль доступной во всех базах данных. Для этого нужно использовать команду CREATE ROLE с опцией LOGIN и указать логин, который должен быть связан с ролью: «CREATE ROLE user\_name LOGIN LOGIN\_NAME login\_name;».

Создав роли для пользователей, нужно определить некоторые права, позволяющие манипулировать БД. Для этого в PostgreSQL предусмотрена возможность выдачи прав как отдельным пользователям, так и ролям. Для выдачи права используется конструкция: «GRANT action\_name ON object\_name TO user\_name/role\_name;», в эту конструкцию можно добавить свойство «WITH GRANT OPTION», позволяющее предоставлять права другим пользователям или ролям. Здесь action\_name - право доступа, которое нужно предоставить (например, SELECT, INSERT, UPDATE или DELETE), object\_name - имя объекта базы данных, к которому нужно предоставить доступ, user\_name или role\_name - имя пользователя или роли, которым нужно предоставить доступ. Для запрета на определенное право используется конструкция: «REVOKE action\_name ON object\_name FROM user\_name/role\_name;», при запрете права также может использоваться «CASCADE», это свойство наоборот запрещает некоторые права тем субъектам, которым оно выдавалось. Помимо выдачи и запрета прав есть третья возможность – отмена прав, она имеет следующий синтаксис: «REVOKE action\_name ON object\_name FROM user\_name/role\_name [CASCADE];», свойство «CASCADE» используется для отмены прав у тех субъектов, которым данный пользователь выдал. Стоит обратить внимание, что для использования команды REVOKE в PostgreSQL необходимо иметь соответствующие привилегии администратора базы данных.



\subsection{Шифрование базы данных}

Вторым базовым механизмом защиты, реализующимся в SQL Server, от НСД является шифрование БД. Шифрование – это преобразование исходной информацию в данные, которые теряют иной смысл, не зная секретного слова или алгоритма, преобразовывающие данные обратно в смысловую информацию. Самый популярный способ зашифровать информацию, хранящуюся в БД – это прозрачное шифрование данных. Преимущество такого вида шифрования заключается в том, что конечные пользователи могут даже и не знать, что применяется защита информации. 
Перед тем, как перейти к шифрованию, следует, на всякий случай, сделать резервную копию БД. Для этого следует включить pgAgen.

Затем выбираем нужную БД, вызываем контекстное меню, в нем выбираем опцию «создать резервную копию».

Создав резервную копию БД, можно приступать к процессу шифрования. Этот процесс происходит в несколько этапов:

\begin{itemize}
    \item создание главного ключа БД,
    \item создание сертификата,
    \item создание ключа шифрования,
    \item запуск процесса шифрования,
    \item проверка состояния шифрования.
\end{itemize}

Главный ключ БД – это ключ, который применяется для шифрования других ключей, использующихся для реализации шифрования в БД. Для его создания применяется следующая конструкция: «CREATE MASTER KEY ENCRYPTION BY PASSWORD 'password';».

Следующим шагом будет создание сертификата. Сертификат – это объект безопасности, имеющий подпись. Этот объект хранит в себе ключи шифрования. Для создания данного объекта применяется следующая конструкция: «CREATE CERTIFICATE name\_db\_and\_cert WITH SUBJECT 'description';».

Последним подготовительным этапом перед началом шифрования БД является создание ключа шифрования. Именно этим ключом будет происходить шифрование БД, а сам ключ будет находиться в сертификате, созданном заранее. Для его создания требуется применять следующий шаблон: «CREATE DATABASE ENCRYPTION KEY WITH ALGORITHM = AES\_128 ENCRYPTION BY SERVER CERTIFICATE cert\_name;».

Следующий шаг является необязательным, но его желательно выполнять. При создании главного ключа, сертификата и ключа шифрования рекомендуется создание резервных копий этих объектов безопасности. Для создания резервной копии ключа шифрования можно использовать команду pg\_dump, которая создаст резервную копию всей базы данных, включая ключи шифрования: «pg\_dump dbname > backup\_file\_name;».

Все объекты безопасности проинициализированы, также для критических объектов были созданы резервные копии. Теперь можно приступать к шифрованию БД. Для начала данного процесса используется следующий пакет команд: «UPDATE table\_name SET column\_name = pgp\_sym\_encrypt(column\_name, 'key\_password');».

Осталось лишь проверить, работает ли ключ шифрования для БД. Для этого используются системные БД «pg\_database\_encryption». Нас интересует столбец is\_encrypted, если шифрование завершилось успешно, то столбец принимает значение 1.




\subsection{Порты}

Порт – это некоторое идентифицирующее число от 1 до 65535, позволяющее протоколам взаимодействовать друг с другом на транспортном уровне модели OSI.

На самом деле, у PostgreSQL сервера по умолчанию установлен только один порт, и это TCP порт.

PostgreSQL использует TCP/IP для обмена данными между клиентами и сервером. По умолчанию, порт TCP для PostgreSQL установлен на 5432. TCP 5432: database engine, используется для подключения к СУБД. Этот порт можно изменить при настройке сервера \cite{online7}..

Несмотря на то, что UDP не используется для коммуникации между клиентами и сервером PostgreSQL, есть несколько расширений, таких как pgpool-II, которые могут использовать UDP для взаимодействия между серверами PostgreSQL.

Таким образом, можно сказать, что PostgreSQL сервер по умолчанию использует только один порт TCP для обмена данными между клиентами и сервером.

Хорошей практикой является смена портов на другие, чтобы при случае атаки порты пришлось перебирать методом грубого подбора, а не использовать установленные по умолчанию. По данным IANA, порты 834-846 являются свободными, поэтому мы их можем использовать.

Для изменения порта, используемого PostgreSQL, можно внести изменения в конфигурационный файл postgresql.conf, который находится в каталоге данных PostgreSQL.

В файле postgresql.conf необходимо изменить параметр port на желаемое значение порта. Например, если вы хотите использовать порт 840 вместо стандартного порта 5432, необходимо изменить строку: port = 5432 на port = 840. После этого нужно перезапустить PostgreSQL, чтобы изменения вступили в силу.

Также можно использовать параметр командной строки -p при запуске сервера PostgreSQL для указания порта. Например, чтобы запустить PostgreSQL на порту 840, необходимо использовать команду: postgres -p 840, но в этом случае необходимо убедиться, что порт 840 свободен и не используется другим приложением на сервере.

Помимо смены порта, слушающий входящие соединения, можно отключить PostgreSQL Server. PostgreSQL Server – это одна из служб PostgreSQL, отвечающая за прослушивание запросов. Если эту службу отменить, то для установления соединения придется явно указывать номер порта, что обеспечивает дополнительную безопасность. Для остановки этой службы требуется просто ввести sudo systemctl stop postgresql в командной строке.



\subsection{Брандмауэр}

Брандмауэр – это программный межсетевой экран, имеющийся в ОС Windows. В свою очередь межсетевой экран – это элемент локальной сети, осуществляющий мониторинг активности в данной сети, а именно фильтрацию сетевого трафика по заранее описанным правилам. PostgreSQL Server функционирует на ОС Windows, поэтому рекомендуется использовать Брандмауэр вместе с программно-аппаратным межсетевым экраном. Такое сочетание позволяет повысить уровень защищенности за счёт использования дополнительных механизмов защиты информации. Также Брандмауэр обеспечивает не только защищенность от НСД, но и от поступающего вредоносного трафика в принципе.

Как было описано выше, межсетевой экран анализирует сетевой трафик и отклоняет его, если он считается подозрительным. Выше мы поменяли стандартный TCP/UDP порт 5432 на 840, теперь нужно объявить правило в межсетевом экране, что публичная сеть не может ссылаться на этот порт. Для этого заходим в Монитор Брандмауэра Защитника Windows в режиме повышенной безопасности и создадим новое правило. Далее определяем, что правило создается для порта, затем вписываем наш нужный порт, после этого разрешаем подключение, после этого исключаем публичный доступ, и в конце сохраняем наше правило.




\textbf{Выводы по разделу}

В этом разделе разбирались аспекты, связанные с безопасностью. При помощи прозрачного шифрования файлы БД были защищены от кражи физических носителей. Создание и управление регистрационных имен позволило осуществить легитимный доступ к информации. А изменение портов и использование Брандмауэра позволило защититься на сетевом уровне. В совокупности была выполнена задача определения защитных мер БД.