\introduction % Структурный элемент: ВВЕДЕНИЕ

За последние несколько лет мы можем наблюдать невероятную скорость развития цифровых технологий в мире. Они оказывают сильное влияние практически на научную деятельность, исследования, производство, транспорт, образование, развлечения и другие сферы. Из-за настолько широкой распространенности, цифровые технологии начинают использоваться государствами для осуществления своих функций и контроля процессов на местах.

Так в России выбрана новая государственная политика цифровизации, которая нацелена на полных переход к автоматизированному режиму работы. Ее последствиями будут различные улучшения, включая улучшение качества образования, социального обеспечения, медицинской помощи и других государственных услуг. С помощью цифровых технологий государственные органы смогут расширить возможности реализации социальных функций государства, увеличить прозрачность своих процессов принятия решений и гражданского контроля процессов управления, а также повысить результативность своей работы и сделать ее более нацеленной на нужды населения.

В медицине цифровые технологии могут сыграть важную роль в повышении качества медицинской помощи и улучшении доступности лечения. Они могут также способствовать более эффективной организации квалифицированной помощи, облегчению процессов диагностики, лечения и ухода за пациентами, а также повышению уровня безопасности и качества медицинского обслуживания.

Еще в конце 20 века основной целью информатизации (так как термин <<цифровизации>> появился позднее)  являлось сокращение времени, затрачиваемого на получение и оказание медицинской услуги. А после утверждения <<Концепции создания ЕГИСЗ>> приказом Министерства здравоохранения и социального развития Российской Федерации от 28 апреля 2011 № 364 <<Об утверждении Концепции создания единой государственной информационной системы в сфере здравоохранения>> был запущен процесс цифровизации здравоохранения в России.

Сейчас происходит поиск решений таких проблем как например маршрутизация пациентов, разработка в области искусственного интеллекта и ее интеграция в работу медиков, а также реорганизация оказания медицинских услуг. Например, во время пандемии COVID-19, во многих больницах и поликлиниках обнаружилась острая нехватка качественного КТ-оборудования и специалистов, которые могут делать описание исследований, полученных с них. Поэтому в Москве 29 апреля 2020 года Сергеем Собяниным был открыт экспертный референс-центр, благодаря которому за два года работы был дистанционно описан 1 миллион 800 тысяч исследований, полученных от поликлиник в виде цифровых рентгеновских снимков. Подобный центр позволяет в 5 раз сократить время, затрачиваемое на описание проведенных исследований, а само описание автоматически попадает в электронную медицинскую карту москвича, доступ к которой имеет как пациент, так его лечащий врач.

Также для решения ряда проблем, связанных с получением и оказанием медицинский услуг в поликлинниках и больницами, в Москве используется крупнейший сервис ЕМИАС, в функционал которого входит система медицинского документооборота, электронные версии карт пациентов, очереди к врачу и рецептов, дистанционное управления направлениями и записями к специалистам.

Пандемия COVID-19 послужила доказательным примером того, как использование цифровых технологий в области медицины могут способствовать росту эффективности ее функционирования, в первую очередь за счет масштабируемости охватов медицинской помощи и, как следствие, возможности своевременного ее оказания.

Но стоит отметить, что на данный момент большинство цифровых государственных решений направлены на работу врачей в пределах стен медицинских учреждений.  А специалисты, работающие в бригадах скорой помощи, по сей день вынуждены сталкиваться с рядом нерешенных проблем.

Например, ручное заполнение бумажной карты вызова. Помимо того, что подобное оформление занимает в разы больше времени, нежели набивание того же текста с помощью клавиатуры, встает вопрос о приведении этого медицинского юридического документа к единой форме на всей территории России. Для этого был создан и утвержден приказом Минздравсоцразвития России от 02.12.2009г. №942 <<Об утверждении статистического инструментария станции (отделения), больницы скорой медицинской помощи>>  определенный стандарт заполнения. Также возможны требования приказов по Станции, которые тоже надо учитывать. Каждая карта вызова проходит проверку главврачом на предмет соответствия стандартам и требованиям. В случае обнаружения нарушений происходит переписывание карты с прилагающимся объяснение причины повторного заполнения. Для того, чтобы уменьшить количество ошибок, связанных с государственными стандартами заполнения этого документа, подстанция может составить собственный шаблон карты, который врачи будут использовать в качестве основы. Однако это происходит не везде. Исходя из всего вышеописанного, встает также вопрос физического хранения такого объема карт вызовов и объяснительных.

Также существует серьезная проблема влияния человеческого фактора на качество оказания медицинской помощи в силу иной специфики работы врача или забывчивости в результате сильного переутомления или стресса организма. Например, нередко бывает такое, что из-за особенности координации карет скорой помощи на вызов к ребенку может приехать взрослая линейная бригада. И если алгоритмы оказания медицинской помощи взрослым практически не отличаются от алгоритмов помощи детям, то в случае препаратов все не так однозначно. Риск передозировки лекарственным средством крайне велик. Поэтому во избежании неправильного лечения многие медики возят с собой бумажные или электронные версии алгоритмов оказания помощи, а также таблицы с детскими дозировками препаратов.

В рамках данной работы будет реализован виджет, предоставляющий информацию о рекомендуемых дозировках лекарственных средств, которыми оснащена бригада скорой помощи, а также о противопоказания к их применению. Актуальность данной темы связана с тем, что данный функционал может частично убрать человеческий фактор, который оказывает негативное влияние на качестве оказанного медицинского обслуживания, из работы медика – специалист всегда будет иметь подстраховку в виде возможности быстрого получения полной актуальной информации по препарату, что может предупредить потенциальные ошибки в ходе оказания помощи. 

Таким образом, выполненная работа актуальна и с научно-методической, и с практической точек зрения.

Объектом разработки в данной работе являются лекарственные средства, их противопоказания и факторы, влияющие на их дозировки. 

Цель работы – разработать и интегрировать в web-приложение ОНМП виджет для расчета дозировок лекарственных средств и выдачи по ним дополнительной информации.

Для достижения поставленной цели в работе были решены следующие задачи:
\begin{itemize}
\item составление функциональных требований к виджету;

\item анализ и структуризация данных о лекарственных средствах, их дозировках и противопоказаниях;

\item разработка моделей данных для хранения информации о препаратах в базе данных и формулировка требований к web-запросам для получения данных о лекарствах;

\item создание макета web-страницы виджета и реализация в соответствии с ним интерфейса;

\item интеграция web-страницы виджета в основное web-приложение ОНМП и настройка ее взаимодействия  с удаленным сервером для получения данных.
\end{itemize}

Работа основывалась на следующих инструментах и методах:
\begin{itemize}
\item для реализации интерфейса клиентской части приложения ОНМП был использован фреймворк React, менеджер пакетов npm, сборщик модулей Webpack, 

\item для взаимодействия клиентской части приложения ОНМП с удаленным сервером были использованы HTTP-запросы и библиотека Axios для асинхронных запросов.
\end{itemize}

Основными результатами, полученными в работе, являются:
\begin{itemize}

\item web-страница виджета <<Калькулятор дозировок>>, реализованная в соответствии с макетом и функциональными требованиями;

\item виджет успешно интегрирован в функционал web-приложения ОНМП;

\item информация о дозировках лекарственных средств и их противопоказаниям, отображаемая на web-странице, актуальна и соответствует стандартам.
\end{itemize}

Результаты работы предназначены для интегрирования их в качестве общего обязательного программного обеспечения на каретах скорой помощи на территории России.

Благодаря использованию такого приложения как ОНМП появляется возможность ускорения заполнения карт, а также частичного устранения человеческого фактора в виде забывчивости и специфики врачебного профиля из работы врачей на скорой помощи за счет виджетов-подсказок  с полной и актуальной информацией, что приведет к улучшению качества медицинского обслуживания в России.


% Цитирование источника 1 \cite{Wikipedia1}.
