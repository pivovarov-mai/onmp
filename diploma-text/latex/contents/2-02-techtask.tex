\section{АНАЛИЗ РЕЗУЛЬТАТОВ РЕАЛИЗАЦИИ КАЛЬКУЛЯТОРА ДОЗИРОВОК И ЕГО ИНТЕГРАЦИИ В WEB-ПРИЛОЖЕНИЕ ОНМП}
\subsection{Тестирование разработанного виджета <<Калькулятора дозировок>>}
В процессе тестирования виджета было выполнено два этапа. Первый этап заключался в проверке соответствия реализованного интерфейса макету, который был запланирован изначально. Данный этап прошел успешно.

Второй этап был реализован на счет unit-тестирования.

Unit-тестирование -- это вид тестирования, при котором каждый модуль или отдельный компонент кода проверяется на соответствие заранее определенным требованиям. В процессе unit-тестирования тестируется функциональность отдельных блоков исполняемого кода, чтобы убедиться, что они работают корректно и без ошибок в изолированных условиях. Unit-тестирование является важной частью процесса разработки программного обеспечения, поскольку позволяет выявить и исправить ошибки на ранних этапах разработки. Кроме того, этот метод позволяет автоматизировать процесс тестирования и ускорить процесс разработки.

Для виджета было написано несколько unit-тестов на соответствие модулей и функций, в том числе поисковой функции по названию препарата и функции сохранения списка лекарственных средств. Эти тесты также были пройдены успешно.

\subsection{Анализ результата разработки}
В результате данной работы был разработан и интегрирован в web-приложение ОНМП виджет калькулятора дозировок. Из особенностей его функционала можно отметить:
\begin{itemize} 

    \item возможность поиска препаратов по названию в общем списке лекарственных средств, которым оснащена бригада;

    \item  возможность редактирования списка препаратов, которые заносятся в карту вызова, а также регулирования дозировки выбранных лекарственных средств;

    \item в карте вызова сохраняются препараты в родительном падеже с указанием дозы, которая была использована во время вызова;

    \item в карту вызова можно добавлять только те препараты, которые числятся в составе лекарственных средств, находящихся в наличии у бригады скорой помощи, что исключает возможность использования препаратов, которые официально не числятся в составе бригады;

    \item для каджого препарата предоставляется список диагнозов, при которых, согласно алгоритмам оказания помощи, возможно использование данное лекарства, с указанием рекомендуемых дозировок;

    \item предоставлена дополнительная информация по препаратам в виде списка противопоказаний, а для детских дозировок указано, каким образом была получена доза;

    \item  функциональность не создает большой нагрузки на backend-сервер благодаря особенностям реализации взаимодействия с удаленным сервером.
\end{itemize}
    
Среди недостатков приложения можно упомянуть, что при возникновении ситуации, когда врачу неудобно будет использовать клавиатуру для поиска препаратов в поисковой строке, у медика не будет возможности воспользоваться альтернативным поиском, например, за счёт голосового помощника.

Также нет возможности запоминать диагноза, основываясь на котором медик дал пациенту то или иное лекарство в сохраненной в карте вызова дозировке. Это может существенно затруднять работу медика, если ему нужно будет перепроверить корректность оказанной помощи и ее соответсвия стандартам.

Еще одна проблема текущего решения связана с тем, что список лекарственных средств подходит только для взрослых бригад неотложной скорой помощи. Это ограничение делает невозможным полноценное использование приложения медиками в других типах бригад. Это происходит из-за того, что лекарственное оснащение бригад зависит от их типа. Для решения этой проблемы требуется расширение списка лекарственных средств для других типов бригад или добавление возможности выбора типа бригады в приложении.

Дополнительно стоит упомянуть, что использование виджета <<Калькулятора дозировок>> в отрыве от web-приложения становится бесполезным, так как часть его функционала, как сохранение списка использованных во время вызова препаратов в карту, перестанет работать без подключения к web-приложению ОНМП.

Данный виджет, который интегрирован в web-приложение ОНМП, может использоваться в качестве основного программного обеспечения для работы взрослых бригад неотложной скорой помощи, чтобы ускорить заполнение карт вызова. Его функционал позволяет медикам быстро добавлять и редактировать информацию об использованных во время вызова препаратах в медицинскую карту вызова, что экономит время. Кроме того, использование виджета помогает сократить влияние человеческого фактора на работу медиков, обеспечивая им более простой доступ к информации о препаратах и их дополнительной информации.
