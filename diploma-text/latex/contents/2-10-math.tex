% \section{МАТЕМАТИКА: ТЕОРЕМЫ, ПРИМЕРЫ, ОПРЕДЕЛЕНИЯ И ЛЕММЫ}

% \begin{definition}\label{dfn1}
%     Это определение и оно нумеруется сквозной нумерацией по всему документу.
% \end{definition}

% \begin{theorem}\label{th1}
%     Это теорема и она также имеет сквозную нумерацию. Ссылка на определение~\ref{dfn1}.
% \end{theorem}

% \begin{proof}
%     Это доказательство.
% \end{proof}

% \begin{corollary}\label{cor1}
%     Следствие имеет нумерацию в пределах одной теоремы. Ссылка на теорему~\ref{th1}.
% \end{corollary}

% \begin{example}\label{ex1}
%     Пример также можно приводить в стиле теоремы. Нумерация сквозная. Ссылка на следствие~\ref{cor1}.
% \end{example}

% Ссылка на пример \ref{ex1}.