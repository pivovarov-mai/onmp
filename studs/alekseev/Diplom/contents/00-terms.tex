\newglossaryentry{id1}{ % Нужны разные id, можно ставить просто последовательно
    name={Модель OSI},
    description={это сетевая модель стека сетевых протоколов, посредством которой различные сетевые устройства могут взаимодействовать друг с другом}
}
\newglossaryentry{id2}{
    name={Наряд},
    description={это группа медицинских работников и транспортных средств, назначенных для выполнения определенной миссии или задачи, а также медицинское оборудование и инструменты}
}\newglossaryentry{id3}{
    name={ACID},
    description={это набор свойств транзакций базы данных, предназначенных для гарантии достоверности данных, несмотря на ошибки, сбои питания и другие неудачи}
}
\newglossaryentry{id4}{
    name={BSON},
    description={это формат электронного обмена цифровыми данными, бинарная форма представления простых структур данных и ассоциативных массивов. Является надмножеством JSON, включая дополнительно регулярные выражения, двоичные данные и даты}
}

\newglossaryentry{id5}{
    name={JSON},
    description={это популярный формат текстовых данных, который используется для обмена данными в современных веб - и мобильных приложениях}
}
\newglossaryentry{id6}{
    name={SQL-инъекция},
    description={это уязвимость, которая позволяет атакующему использовать фрагмент вредоносного кода на языке структурированных запросов (SQL) для манипулирования базой данных и получения доступа к потенциально ценной информации}
}
\newglossaryentry{id7}{
    name={SSL},
    description={это криптографический протокол, который подразумевает более безопасную связь и использует асимметричную криптографию для аутентификации ключей обмена, симметричное шифрование для сохранения конфиденциальности, коды аутентификации сообщений для целостности сообщений}
}
\newglossaryentry{id8}{
    name={TCP},
    description={это важный протокол сети интернет, который позволяет двум хостам создать соединение и обмениваться потоками данных}
}
\newglossaryentry{id9}{
    name={UDP},
    description={это один из ключевых элементов набора сетевых протоколов для Интернета, с помощью которого компьютерные приложения могут посылать сообщения другим хостам по IP-сети без необходимости предварительного сообщения для установки специальных каналов передачи или путей данных}
}