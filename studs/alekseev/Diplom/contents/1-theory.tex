% Название разделов -- все прописные
\section{ТЕОРЕТИЧЕСКОЕ ОБОСНОВАНИЕ РАЗРАБОТКИ БАЗЫ ДАННЫХ}

\subsection{Анализ предметной области}

Анализ предметной области является первым этапом в проектировании базы данных, в ходе которого выделяются основные объекты и их свойства, определяются первоначальные требования и границы проекта, чтобы разработать эффективную и безопасную базу данных.

Предметной областью разрабатываемой модели данных является система заполнения электронных медицинских карт для отделения неотложной медицинской помощи.

Скорая медицинская помощь (СМП) – вид медицинской помощи, оказываемой гражданам при заболеваниях, несчастных случаях, травмах, отравлениях и других состояниях, требующих срочного медицинского вмешательства.

Отделение неотложной (экстренной) медицинской помощи больницы является структурным подразделением многопрофильной больницы, которое в круглосуточном режиме оказывает экстренную (неотложную) медицинскую помощь.

Медицинская помощь включает в себя процедуру проведения осмотра, непосредственно манипуляции по оказанию медицинской помощи, а также консультирование пациента по с целью определения наиболее эффективного, безопасного и экономически оправданного курса лечения.

ОНМП осуществляет следующие функции:

\begin{itemize}
    \item прием пациентов с острыми заболеваниями, несчастными случаями, травмами, отравлениями и другими состояниями, требующими немедленной медицинской помощи;
    \item проведение первичной медицинской диагностики и оценки состояния пациента, осуществление мер, направленных на стабилизацию состояния пациента;
    \item предоставление неотложной медицинской помощи, включая проведение лечебных манипуляций, инъекций, переливание крови, а также оказание психологической помощи;
    \item организация и координация работы других специалистов и служб в медицинском учреждении, если необходимо;
    \item подготовка пациента к транспортировке в стационар для дальнейшего лечения;
    \item соблюдение всех медицинских стандартов и требований, направленных на обеспечение безопасности пациентов и медицинского персонала;
    \item организация транспортировки пациентов в случае необходимости;
    \item обеспечение работы необходимого медицинского оборудования, инструментов, материалов и медикаментов, необходимых для оказания неотложной медицинской помощи,
    \item проведение медицинской документации, включая учет медицинских случаев, регистрацию историй болезни и медицинских записей;
    \item обеспечение мониторинга и контроля за состоянием пациентов, находящихся на лечении в отделении неотложной (экстренной) медицинской помощи;
    \item обучение медицинского персонала и сотрудников отделения неотложной медицинской помощи новым методам лечения и диагностики;
    \item сотрудничество и консультации с другими медицинскими учреждениями, специалистами и службами для повышения качества оказываемой медицинской помощи.
\end{itemize}

Отделение неотложной (экстренной) медицинской помощи является ключевым звеном в системе оказания медицинской помощи населению и обеспечивает быстрое и эффективное лечение при острых заболеваниях и травмах.

Базы данных для неотложной медицинской помощи являются критически важным элементом в работе современных медицинских учреждений. Они позволяют эффективно организовывать, хранить и обрабатывать медицинскую информацию, ускоряя процессы принятия решений и повышая качество медицинской помощи.

Основным требованием к базе данных для неотложной медицинской помощи является ее способность оперативно и точно хранить медицинскую информацию о пациентах. Эта информация может включать данные о медицинской истории пациента, диагнозе, принятых мероприятиях, результатах обследований и лекарственном лечении.

Для эффективного управления медицинской информацией в базе данных неотложной медицинской помощи необходима специализированная система управления базами данных (СУБД). Существует множество СУБД, которые могут использоваться для этой цели, включая Oracle, MySQL, Microsoft SQL Server, PostgreSQL и др.

Также необходимо учитывать специфику работы медицинских учреждений, которые могут иметь различные требования к хранению и обработке медицинской информации. Например, больницы могут иметь разные отделения, каждое из которых может иметь свои особенности в обработке информации. Кроме того, необходимо учитывать возможность интеграции с другими системами, такими как системы управления ресурсами и планирования процессов.

Важно также отметить, что база данных для неотложной медицинской помощи должна соответствовать требованиям законодательства в области защиты персональных данных, таких как Федеральный закон от 27 июля 2006 года № 152-ФЗ «О персональных данных». Регулярное обновление и адаптация базы данных к изменениям в законодательстве помогут поддерживать соответствие нормам и предотвращать правовые проблемы.

Исследования показывают, что эффективное использование баз данных в медицинских учреждениях может значительно улучшить качество медицинской помощи и снизить затраты на ее оказание. Например, использование баз данных может уменьшить количество ошибок в принятии решений, сократить время, необходимое для оказания медицинской помощи, и повысить точность диагностики.

На текущий момент существует множество различных подходов к проектированию баз данных для медицинских учреждений. Некоторые из них ориентированы на сохранение структурированной информации, другие на работу с полуструктурированными и неструктурированными данными.

Одним из наиболее распространенных подходов к проектированию баз данных для медицинских учреждений является использование реляционной модели данных \cite{book6}. Реляционная модель данных позволяет хранить информацию в виде таблиц, которые могут быть связаны друг с другом через ключи. Это позволяет обрабатывать большие объемы структурированных данных и осуществлять запросы на выборку данных, используя язык SQL.

Тем не менее, в последние годы набирают популярность нереляционные базы данных, такие как MongoDB, Cassandra, Couchbase и др. Они хранят данные в более гибкой форме, позволяют более легко масштабировать базы данных и работать с полуструктурированными и неструктурированными данными, такими как тексты медицинских записей, изображения и видео.

Важным аспектом разработки баз данных для медицинских учреждений является также обеспечение защиты медицинской информации от несанкционированного доступа. Для этого применяются различные меры безопасности, такие как шифрование данных, многофакторная аутентификация и разграничение прав доступа к информации в зависимости от роли пользователя.

Наконец, важно отметить, что разработка баз данных для медицинских учреждений является сложным и многогранным процессом, требующим учета множества факторов, таких как специфика медицинских процедур, законодательство, требования к безопасности и другие. Поэтому необходимо проводить тщательный анализ предметной области и разрабатывать индивидуальные решения для каждого конкретного медицинского учреждения.




\subsection{Анализ существующих подходов}

Перед тем как начинать разработку базы данных, нужно выяснить, какие виды БД существуют, проанализировать их и уже на основе этого делать выводы о наиболее подходящем виде под наши конкретные требования.

Существует несколько видов баз данных, каждый из которых имеет свои особенности и применения:

\begin{itemize}
    \item реляционные базы данных (РБД): это самый распространенный тип БД. В РБД данные организованы в виде таблиц, состоящих из строк (записей) и столбцов (полей). Реляционные БД используют структурированный язык запросов, такой как SQL, для управления данными и выполнения операций, таких как вставка, обновление, удаление и извлечение данных. Примеры реляционных БД включают MySQL, PostgreSQL и Oracle;
    \item иерархические базы данных: в таких БД данные организованы в виде иерархической структуры, состоящей из уровней и подуровней. Каждый уровень может иметь только одного родителя, что создает иерархию. Этот тип БД широко использовался в прошлом, но в настоящее время его применение ограничено. Примеры иерархических БД включают IBM's Information Management System (IMS);
    \item сетевые базы данных: этот тип БД организован в виде сети связанных записей. Записи могут иметь несколько родителей и несколько дочерних записей, что позволяет создавать сложные связи между данными. Сетевые БД также имеют ограниченное применение в настоящее время. Примером сетевой БД является Integrated Data Store (IDS);
    \item объектно-ориентированные базы данных (ООБД): в ООБД данные организованы в виде объектов, которые могут содержать свойства (поля) и методы (функции). ООБД позволяют более натуральное представление сложных структур данных и поддерживают наследование и полиморфизм. Примеры ООБД включают MongoDB и Couchbase;
    \item ключ-значение базы данных: в таких БД данные хранятся в виде пар ключ-значение. Они просты и эффективны для хранения и извлечения данных, но ограничены по функциональности. Примеры ключ-значение БД включают Redis и Apache Cassandra;
    \item документоориентированные базы данных: в этом типе БД данные организованы в виде документов, обычно в формате JSON или XML. Документы могут содержать различные поля и вкладываться друг в друга, образуя более сложную структуру;
    \item временные базы данных: эти базы данных предназначены для хранения и управления временными данными, такими как временные ряды, события и временные маркировки. Они оптимизированы для выполнения операций, связанных с временными данными, такими как агрегация, интерполяция и анализ трендов. Примеры временных баз данных включают InfluxDB и TimescaleDB;
    \item массивные параллельные обработки (MPP) базы данных: эти базы данных разработаны для обработки больших объемов данных с использованием параллельных вычислений на распределенных системах. Они позволяют выполнить параллельную обработку запросов и аналитики на большом количестве узлов. Примеры MPP БД включают Amazon Redshift и Google BigQuery.
\end{itemize}
    
Каждый из этих видов БД предназначен для решения определенных задач и имеет свои преимущества и ограничения. Выбор типа базы данных зависит от конкретных требований проекта, характеристик данных и ожидаемой производительности.

После того, как были рассмотрены различные виды БД, приведем существующие аналоги предполагаемой разработки. На данный момент существует несколько общеизвестных баз данных для отделений неотложной медицинской помощи:

\begin{itemize}
    \item федеральная медицинская информационная система (ФМИС): ФМИС является единым информационным ресурсом для системы здравоохранения в России. Она объединяет различные базы данных и модули, включая модуль неотложной медицинской помощи. В рамках этой системы регистрируются и хранятся данные о поступающих в отделения неотложной помощи пациентах. В ФМИС включены сведения о пациентах, результаты обследований и лечения, информация о медицинских событиях;
    \item единая автоматизированная информационная система "Амбулатория": эта система разработана для работы с амбулаторными картами пациентов и учета медицинской информации. Она может быть использована в поликлиниках и отделениях неотложной помощи. В рамках системы "Амбулатория" регистрируются пациенты, ведется электронная история болезни, хранятся результаты обследований и лечения. Система также позволяет врачам планировать визиты, назначать лекарства и процедуры, а также отслеживать текущее состояние пациента;
    \item система учета вызовов скорой медицинской помощи (СУВ): эта система предназначена для приема и учета вызовов скорой помощи. Она используется диспетчерскими службами скорой помощи для регистрации информации о вызывающем лице, месте происшествия, описании ситуации и оценке состояния пациента. СУВ позволяет оперативно отправлять бригады скорой помощи на место происшествия и отслеживать их движение в реальном времени. В системе также регистрируются данные о времени прибытия бригады, оказанной помощи и транспортировке пациента;
    \item информационная система "Регистр оказания скорой медицинской помощи" (РОСМЕД): эта система используется для сбора и анализа данных о оказании скорой медицинской помощи в России. РОСМЕД предназначена для регистрации информации о вызовах скорой помощи, диагнозах, примененных медицинских манипуляциях и результатах оказания помощи. В систему вносятся данные о каждом вызове скорой помощи, включая информацию о времени вызова, причине вызова, медицинских мерах, принятых бригадой скорой помощи, а также о результате лечения или транспортировке пациента.
\end{itemize}

Эти базы данных - ФМИС, система "Амбулатория", СУВ и РОСМЕД - представляют собой информационные системы, используемые в различных аспектах медицинской помощи в России. Они помогают в сборе, хранении и управлении медицинской информацией, что способствует более эффективному и качественному оказанию неотложной медицинской помощи пациентам.

Все существующие БД хорошо выполняют свой спектр функций, но ни одна из рассмотренных систем не дает автоматизации и упрощения работы медицинского работника непосредственно во время вызова. Работнику все равно приходится работать с бумажными носителями и заполнять все вручную, а лишь потом происходит оцифровка всех собранных данных о пациенте.

Наша же задача стоит в том, чтобы организовать весь спектр проводимых манипуляций с информацией от момента вызова, до момента возвращения бригады в ОНМП в электронном виде для ускорения работ и минимизации ошибок при заполнении медицинских карт, выставлении диагноза и оказания требуемой медицинской помощи.




\subsection{Назначение и возможности базы данных}

В предметной области системы заполнения электронных медицинских карт для отделения неотложной медицинской помощи, основные объекты и свойства, которые следует рассмотреть, могут включать:

\begin{itemize}
    \item пациенты: информация о каждом пациенте, включая персональные данные (имя, дата рождения, пол и контактная информация), медицинскую историю, диагнозы, принятые мероприятия, результаты обследований и лекарственное лечение;
    \item медицинские работники: данные о врачах, медсестрах и других медицинских специалистах, включая их идентификационные данные, специализацию, график работы и доступные привилегии;
    \item медицинские процедуры: информация о проводимых процедурах, включая коды процедур, описания, стоимость, требуемое оборудование и прочие детали;
    \item отделения: данные о различных отделениях неотложной медицинской помощи в больнице, их назначение, доступный персонал и оборудование;
    \item ресурсы: информация о доступных ресурсах, таких как медицинское оборудование, лекарства, материалы и другие необходимые средства;
    \item расписание: график работы медицинского персонала и расписание доступности отделений и ресурсов;
    \item системы управления и интеграция: необходимо учесть возможность интеграции с другими системами, такими как системы управления ресурсами и планирования процессов, чтобы обеспечить эффективное взаимодействие и координацию деятельности медицинских учреждений.
\end{itemize}

Для лучшего понимания предметной области, рассмотрим конкретный пример - разработку базы данных для отделения неотложной медицинской помощи.

Медицинские работники оказывают медицинскую помощь пациентам с острыми заболеваниями и травмами, которые требуют немедленного вмешательства, также необходимо быстро и точно определить диагноз, назначить лечение и принять меры по сохранению жизни пациента.

Одной из основных задач базы данных для отделения неотложной медицинской помощи является хранение и обработка медицинских данных пациентов, включая информацию о симптомах, диагнозах, назначенных лекарствах, процедурах и т.д. Кроме того, необходимо учитывать, что пациенты могут обращаться за медицинской помощью в нескольких отделениях неотложной медицинской помощи, поэтому база данных должна позволять обмениваться информацией между различными медицинскими учреждениями.

Для решения этих задач можно использовать реляционную базу данных, в которой каждый пациент будет представлен в виде отдельной записи в таблице, содержащей данные о пациенте, диагнозах, лекарствах и т.д. Ключами в таблицах могут быть номера пациента, номера записи и т.д. Это позволит производить выборку данных о конкретном пациенте, обращаться к истории его болезни, проводить анализ данных и выявлять тенденции в заболеваемости и лечении.

Однако, реляционная модель может столкнуться с проблемами при работе с полуструктурированными и неструктурированными данными, такими как медицинские изображения, видео и тексты медицинских записей. Для работы с этими данными может использоваться нереляционная база данных, такая как MongoDB. В MongoDB данные могут быть храниться в более гибкой форме, используя форматы, такие как JSON и BSON. MongoDB также позволяет хранить и обрабатывать файлы в формате BLOB (binary large object), что делает его идеальным инструментом для хранения медицинских изображений и других неструктурированных данных.

Еще одной важной задачей при разработке базы данных для отделения неотложной медицинской помощи является обеспечение безопасности хранения и доступа к медицинским данным. Для этого может использоваться различные меры, такие как шифрование данных, авторизация пользователей и аудит доступа.

Также при проектировании базы данных для отделения неотложной медицинской помощи необходимо учитывать требования к ее масштабируемости, отказоустойчивости и производительности. В случае большого количества пациентов и медицинских записей может потребоваться использование кластерной архитектуры, репликации данных и других технологий, позволяющих обеспечить высокую доступность и производительность базы данных. Также необходимо обрабатывать большие объемы данных и поддерживать быстрый доступ к ним \cite{online14}.

На сегодняшний день существует множество различных систем управления базами данных, которые могут быть использованы для разработки базы данных для отделения неотложной медицинской помощи, включая MySQL, PostgreSQL, Oracle Database, Microsoft SQL Server, MongoDB и др. Выбор конкретной системы управления базами данных зависит от требований к производительности, масштабируемости, отказоустойчивости и других факторов.

Анализируя все вышеперечисленные требования к организации базы данных, можно сделать вывод, что основным субъектом данной базы данных является врач, который будет осуществлять выезд и непосредственное оказание всех необходимых медицинских услуг.

Врач может иметь доступ к данным о препаратах, находящихся в использовании у данного наряда, он может осуществить поиск по названию лекарственного средства для оперативного предоставления ответа по запросу клиента.

Основные реализуемые функции:
\begin{itemize}
    \item аутентификация и авторизация;
    \item разграничение ролей пользователей;
    \item добавление и удаление данных;
    \item поиск данных по нескольким критериям;
    \item вывод вспомогательных информационных таблиц.
\end{itemize}

\bigbreak
\textbf{Выводы по разделу}
\bigbreak

Проектирование базы данных для системы заполнения электронных медицинских карт является сложным процессом, требующим тщательного анализа предметной области, определения требований и границ проекта, выбора подходящей модели данных и СУБД, обеспечения безопасности данных, обучения пользователей и поддержки системы. Это комплексный процесс, который должен выполняться с участием экспертов в области медицины и баз данных для достижения оптимального результата. Результатом успешной разработки будет эффективная система управления медицинской информацией, способствующая повышению качества медицинской помощи, оптимизации процессов и улучшению результатов лечения пациентов.

В конечном итоге, разработка базы данных для отделения неотложной медицинской помощи является сложной задачей, которая требует учета множества факторов, таких как безопасность, масштабируемость, отказоустойчивость и производительность. Решение этих задач требует использование различных технологий и систем управления базами данных, а также тщательного анализа требований и потребностей пользователей.