\abstract % Структурный элемент: РЕФЕРАТ

\keywords{БАЗА ДАННЫХ, АВТОМАТИЗАЦИЯ, БЕЗОПАСНОСТЬ, ШИФРОВАНИЕ, POSTGRESQL}

Объектом исследования в данной выпускной квалификационной работе является деятельность отделения неотложной медицинской помощи.

Предметом исследования является процесс разработки и внедрения базы данных медицинских диагностических таблиц первичного осмотра пациента, таблиц дифференциальной диагностики и вспомогательных информационных таблиц.

Целью работы является автоматизация, ускорение и улучшение условий обработки данных для сотрудников отделения неотложной медицинской помощи.

Основное содержание работы состояло в разработке базы данных медицинских диагностических таблиц первичного осмотра пациента для сотрудников отделения неотложной медицинской помощи.

Для достижения поставленной цели был проведен анализ функциональных требований для разработки необходимой базы данных, а также анализ наиболее эффективных способов хранения взаимосвязанных атрибутов каждой сущности. Кроме того, была разработана и протестирована довольно обширная структура базы данных, объединяющая сильные стороны проанализированных методов хранения и обработки информации.

Основными результатами работы, полученными в процессе анализа и разработки являются: определение наиболее эффективных способов структурирования и хранения всей необходимой информации для работников скорой медицинской помощи, разработка и внедрение базы данных, включая вспомогательные информационные таблицы, оценка эффективной работы полученной базы данных и выявление сценариев ее оптимального использования.

Полученные результаты разработки представляют немаловажное значение для всех организаций здравоохранения, стремящихся оптимизировать работу своего персонала и минимизировать время и трудозатраты на рутинное заполнение бумажных документов.