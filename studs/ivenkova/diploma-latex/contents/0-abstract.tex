\abstract % Структурный элемент: РЕФЕРАТ

\keywords{ВЕБ-ПРИЛОЖЕНИЕ, ОНМП, ТАБЛИЦЫ ДИФФЕРЕНЦИАЛЬНОЙ ДИАГНОСТИКИ, REACT}

Объектом разработки в данной работе являются таблицы дифференциальной диагностики, используемые врачами скорой медицинской помощи. 

Целью работы является внедрение данных таблиц в веб-приложение ОНМП (отделение неотложной медицинской помощи) для более быстрого и простого их нахождения, расчёта результата и, если необходимо, его автоматического внесения в медицинскую карту пациента.

Для достижения поставленной цели были проведены исследования существующих таблиц, их типов, структуры. Рассмотрены текущие виды представления таблиц. 

реализованная программа написана на языке JavaScript с помощью библиотеки React: реализована обработка API-запросов к серверу для получения данных из базы данных, перевод таблиц из текстового формата в визуальный, разработана логика выбора пользователем строк или ячеек.

Основным результатом работы, полученным в процессе разработки, является функция веб-приложения ОНМП, позволяющая находить и использовать интерактивные таблицы дифференциальной диагностики для рассчёта дифференциального диагноза. 

Данные результаты разработки предназначены для врачей скорой медицинской помощи.

Применение результатов данной работы позволит врачам скорой помощи быстро находить нужный материал, а также ускорит заполнение медицинской карты путём автоматического рассчёта дифференциального диагноза.
