\introduction % Структурный элемент: ВВЕДЕНИЕ

Актуальность темы данной работы связана с тем, что на данный момент врачи скорой помощи вынуждены носить  с собой огромные папки с различными справочными материалами, которые им могут понадобиться в работе. В эти справочные материалы кроме всего прочего входят таблицы дифференциальной диагностики, помогающие поставить промежуточный диагноз пациенту - например определить площадь ожогов или стпень боли. Так как этих таблиц насчитывается несколько десятков, а то и сотен, то, соответственно запомнить их все очень сложно. Также долго приходится их искать среди всех прочик бумаг врачей. На это уходит драгоценное время, которое в работе врачей скорой помощи может играть решающую роль. Данные таблицы можно найти в интернете, но их огромное количество, нет единого сайта, содержащего их все. Кроме того на данный момент не существует ни одной реализации интерактивной диф. таблицы, которая бы автоматически рассчитывала результат в соотвествии с выбранными данными. Таким образом, выполненная работа актуальна и с теоретической, и с практической точек зрения.

Цель работы – внедрение данных в таблиц в веб-приложение ОНМП для более быстрого и простого их нахождения, рассчёта результата и, если необходимо, его автоматического внесения в медицинскую карту пациента.

Для достижения поставленной цели в работе были решены следующие задачи:

\begin{itemize}
\item Существующие диф. таблицы проанализиролваны и разделены на классы по типу структуры и организиции. В некоторых таблицах чуть изменена структура для того, чтобы они подходили определённым классам. Также разделены на типы диагнозы - результат рассчёта таблицы, который записывается в мед. карту;
\item Выделены те структурные и логические элементы таблиц, которые являются общими для всех и которые надо хранить в базе данных;
\item Изучен язык программирования JavaScript, библиотека React, MUI; 
\item Реализована основа страницы поиска таблиц и страницы работы с конкретной таблицой;
\item Реализованы API-запросы к серверу для функции поиска и для получения конкретной таблицы;
\item реализована программа, переводящая таблицу из текстового формата json-файла в видимое отображение;
\item Проработана и реализована логика выбора пользователем строк или ячеек таблицы, рассчёт на основе этого итогового результата;
\item Добавлена функция, дающая возможность сохранить результат в мед. карте пациента.
\end{itemize}

Работа основывалась на следующих инструментах и методах:
\begin{itemize}
\item Язык программирования JavaScript;
\item библиотека React, основанная на  JavaScript;
\item библиотека для React - MUI;
\item менеджер пактеов npm, собирающий и запускающий проекты на JavaScript.
\end{itemize}

Основными результатами, полученными в работе, являются:
\begin{itemize}
\item Функция поиска таблиц по их названию;
\item Интерактивная работа с таблицами: возможность выбирать одну строку, несколько строк, определённые ячейки в таблице;
\item Суммирование или иное действие с выбранными ячейками или определёнными полями выбранных строк;
\item Рассчёт результата и вывод его в отдельном поле снизу таблицы
\item Кнопки выбора - надо ли сохранять полученный диагноз в мед. карте пациента.
\end{itemize}

Результаты работы предназначены для применения врачами скорой помощи а также любыми другими пользователями для получени справки или постановки дифференциального диагноза. 

Использование разработки позволяет быстро находить нужную таблицу, рассчитать результат по ней и автоматически занести его в мед.карту пациента. Всё это значительно ускоряет и облегчает работу врачей скорой помощи.
