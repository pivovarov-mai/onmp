\conclusion

В результате выполнения выпускной квалификационной работы бакалавра были реализованы:

\begin{itemize}
    \item поиск таблицы по названию;
    \item работа с конкретной выбранной таблицей;
    \item рассчёт результата;
    \item внесение результата в медицинскую карту.
\end{itemize}

Результаты вывода таблиц проверены на верность и корректность.

Работа велась с двадцатью таблицами. Они были разбиты на типы и соответственно была унифицированна работа с ними. Также были структурированны результаты записи в карту. Таблицы также были внедрены в веб-приложение \url{onmp.ru}, предназначенного для заполнения карты вызова скорой помощи.

Данные внедрённые таблицы предназначены для использования врачами скорой и неотложной медицинской помощи.

Они позволят расширить область справочного материала, доступного врачам -- они смогут искать вспомогательные таблицы по самым разным отраслям медицины.  

Также этот поиск будет происходить быстрее, так как все эти таблицы будут собраны в одном месте и поиск выполняется не вручную, а программой. 

Кроме того, данные таблицы позволят программно рассчитывать результат, что снизит вероятность врачебной ошибки по сравнению с тем, когда врач "в уме" вспоминает таблицу или складывает числа.

Также в итоге данные таблицы можно будет использовать в одном приложени вместе с основной функцией -- заполнением карты вызова. Являясь частью веб-приложения, таблицы также могут помочь ускорить работу врачей, чтобы им не приходилось искать информацию среди нескольких разных источников -- всё собрано в одном месте.

Данная программа, хотя и не реализует задачу дифференциальной диагностики, но помогает врачам скорой помощи быстрее рассчитывать показатели пациентов, что значительно облегчает их работу. Программа может быть улучшена за счёт использования более сложных структур таблиц дифференциальной диагностики с помощью разрабатываемого сейчас модуля S2 библиотеки Ant Design, позволяющего реализовать множество вспомогательеых функций в электронных таблицах. Также может быть реализована функция сохранения состояния таблиц на время всего существования карты. Кроме того, удобным будет создание в будущем отдельного функционала, позволяющего использовать данные таблицы в отрыве от карты и без регистраци пользователя -- ради более быстрого доступа к ним как к вспомогательному материалу.