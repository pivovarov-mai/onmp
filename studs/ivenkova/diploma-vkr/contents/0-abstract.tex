\abstract % Структурный элемент: РЕФЕРАТ

\keywords{ВЕБ-ПРИЛОЖЕНИЕ, ОНМП, ТАБЛИЦЫ ДИФ.ДИАГНОСТИКИ, REACT}

Объектом разработки в данной работе являются таблицы дифференциальной диагностики, используемые врачами скорой медицинской помощи. 

Целью работы является внедрение данных в таблиц в веб-приложение ОНМП для более быстрого и простого их нахождения, рассчёта результата и, если необходимо, его автоматического внесения в медицинскую карту пациента.

Для достижения поставленной цели были проведены исследования существующих таблиц, их типов, структуры. Рассмотрены текущие виды представления таблиц. 

реализованная программа написана на языках React и JavaScript: реализована обработка API-запросов к серверу, перевод таблиц из текстового формата в визуальный, разработана логика выбора пользователем строк или ячеек.

Основным результатом работы, полученным в процессе разработки, является функция веб-приложения ОНМП, позволяющая находить и использовать интерактивные таблицы диф. диагностики для рассчёта диф. диагноза. 

Данные результаты разработки предназначены для врачей скорой медицинской помощи.

Применение результатов данной работы позволит врачам скорой помощи быстро находить нужный материал, а также ускорит заполнение мед.карты путём автоматического рассчёта диф.диагноза.