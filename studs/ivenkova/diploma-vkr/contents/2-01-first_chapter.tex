% Название разделов -- все прописные
\section{ПОСТАНОВКА ЗАДАЧИ}

\subsection{Потребность в разработке интерактивных таблиц дифференциальной диагностики}

Дифференциальная диагностика – это процесс, при котором врач проводит различие между двумя или более заболеваниями, которые могут вызвать наблюдаемые у человека симптомы.

Зачастую не существует лабораторных методов, которые могли бы окончательно поставить причину симптомов заболевания. Это происходит потому, что многие состояния имеют одинаковые или сходные симптомы, а некоторые проявляются по-разному. Чтобы поставить диагноз, врач может использовать метод, называемый дифференциальной диагностикой.

Дифференциальная диагностика включает в себя составление списка возможных состояний, которые могут быть причиной симптомов. Врач будет основываться на информации, которую он получает от:

\begin{itemize}
    \item истории болезни человека
    \item результатов физического обследования
    \item диагностического тестирования
\end{itemize}

Именно значения, полученные одним или несколькими из данных способов лежат в основе таблиц дифференциальной диагностики. Сопоставляя значения, таблицы могут помочь определить, болен ли человек конкретным заболеванием или нет; они определяют его степень, выраженность, стадию, могут рассчитать объём поражения организма. Также таблицы могут подсказать нормы показателей для разных случаев или возрастов пациентов.

Таблицы представляют собой компактное представление набора симптомов и соответствующего им заболевания. Это очень удобный вид представления информации, так как он помогает вычленить основные моменты и быстрее определить диагноз. Врачам не надо долго искать нужную информацию в учебниках, книгах, или статьях. В связи с чем очень многие врачи (а также студенты или обычные люди) часто пользуются ими для определения чего-либо. Пример таблицы дифференциальной диагностики приведён на рисунке~\ref{fig:tab1}.

\begin{figure}
  \includegraphics[scale=0.6]{src/ВАШ. НОШ. Шкалы оценки интенсивности боли.jpg}
  \caption{Пример таблицы дифференциальной диагностики}
  \label{fig:tab1}
\end{figure}

Но на данный момент врачи скорой помощи вынуждены носить с собой огромные папки с различными справочными материалами, которые им могут понадобиться в работе, в их числе и таблицы дифференциальной диагностики. В целом данных таблиц существует несколько сотен -- в самых разных отраслях медицины. Физически переносить такой объём материала, очевидно, очень трудно, поэтому врачи скорой помощи носят с собой только самые основные и часто используемые. В остальном им приходится полагаться на свою память. Но даже при таком объёме поиск нужного материала среди них занимает довольно продолжительное время, а время в работе врачей скорой помощи может играть решающую роль. 

Также в целом у всех врачей значительную часть времени отнимает заполнение медицинской карты пациента.

В связи с этим остро стоит потребность цифровизации таблиц дифференциальной диагностики.

Во-первых, это снизило бы физическую нагрузку на врачей, так как у них отпала бы необходимость в перенеске объёмных бумажных материалов. Во-вторых, это снизило бы и психологическую нагрузку -- ведь у врачей скорой помощи очень напряжённая работа, и к этому ещё прибавляется беспокойство о том, правильно ли они запомнили показатели той или иной болезни, не забыли ли они чего-либо. С введением цифровых таблиц дифференциальной диагностики  часть этой нагрузки уйдёт. В-третьих, они также позволят снизить процент неправильной постановки диагноза, так как врач сможет найти нужную информацию в данных таблицах. Кроме того в приложении может быть доступ ко многом сотням таблиц, что расширит возможности врачей. И, наконец, это ускорит работу врачей в целом - как и при поиске нужного материала, так и при заполнении медицинской карты пациента, так как разультат можно автоматически занести в карту.

\subsection{Анализ существующих подходов к решению проблемы}

Существуют разлицные представления таблиц дифференциальной диагностики. Первый - это бумажный вариант, котрый, как показано выше, имеет множество недостатков.

Существуют различные сайты в интернете, где приведены некоторые таблицы дифференциальной диагностики. Пример --  сайт \url{https://medcollege-old.bsu.edu.ru/uchebniki/tabldif.htm}, показанный на рисунке~\ref{fig:tab2}\cite{medcollege}.

\begin{figure}
  \includegraphics[scale=0.5]{src/extab1.png}
  \caption{Пример сайта с некоторыми таблицами дифференциальной диагностики}
  \label{fig:tab2}
\end{figure}

Но у него есть существенный недостаток, как впрочем и у остальных подобных сайтов в интернете: в них представлена лишь часть таблиц дифференциальной диагностики, набранных или случайным образом, или в соответствии с тематикой сайта, например "Желудочные заболевания у детей". То есть нет единого сайта с таблицами дифференциальной диагностики, который бы объединял разные таблицы по разным направлениям -- при использовании сети интернет приходится по одиночке искать нужную таблицу, что также отнимает время. И не всегда таблицу можно найти -- она может быть вообще не представлена в электронном виде, или же существовать лишь в виде нескольких абзацев текста.

Существует также чат-бот помощник в Telegram под названием "PARAMEDIC (шпаргалки)" \cite{paramedic}, который предназначен специально для помощи работникам скорой медицинской помощи (изображён на рисунке~\ref{fig:bot}.а).  Он выдаёт справочные материалы на самые различные темы во меногих отраслях медицины. В том числе в нём есть примеры заполненных медицинских карт, а также множество схем и таблиц дифференциальной диагностики. Его преимущество перед сайтами из интернета заключается в обширности разнообразного материала, систематезированного по категориям. Минус же данного бота в том, что в нём нет поиска по названию --  пользователю приходится переходить несколько раз по множеству папок, ища нужное. Кроме того, сего существенных недостаток заключается в том, что данные в нём пердставлены в совершенно различных форматах -- таблицы и схемы могут быть в виде картинок, файлов с раширением .pdf, в виде ссылки на Яндекс-диск, где хранится множество файлов или же ссылки на другой сайт со статьей по данной теме (рисунок~\ref{fig:bot}.б). То есть нет единого формата хранения, а пользователь может переходить по совершенно разным ссылкам на другие ресурсы. 

\begin{figure}
  \begin{subfigure}[t]{0.1\linewidth}
    \centering
    \includegraphics[scale=0.4]{src/exbot1.png}
    \caption{}
  \end{subfigure}
  \hfill
  \begin{subfigure}[t]{0.5\linewidth}
    \centering
    \includegraphics[scale=0.4]{src/exbot2.png}
    \caption{}
  \end{subfigure}
  \caption{Чат-бот "PARAMEDIC (шпаргалки)"}
  \label{fig:bot}
\end{figure}

Также существуют специальные программы, предназначенные для дифференциальной диагностики пациентов.

Одна из таких программ - DiagnosisPro \cite{DiagnosisPro}, представлена на рисунке~\ref{fig:diagpro}. Она была разработана специально для повышения качества медицинской помощи и предотвращения диагностических ошибок.

\begin{figure}
  \includegraphics[scale=0.5]{src/diag_pro6.jpg}
  \caption{Программа DiagnosisPro 6.0}
  \label{fig:diagpro}
\end{figure}

Эта программа была очень распространена в США и могла выдавать около 15 000 проявлений заболеваний, таких как симптомы, лабораторные исследования, ЭКГ, рентгеновские снимки, снимки компьютерной томографии, МРТ, УЗИ, патологии, результаты микробиологии и многое другое. 

Как можно понять, она в том числе частично реализовывала функцию таблиц дифференциальной диагностики -- выдавала список симптомов болезней. Почему только частично -- потому, что основная цель этой программы немного отличалась от целей таблиц дифференциальной диагностики. Программа позволяла составить список возможных заболеваний и потом постепенно исключить из него неподходящие варианты. Тогда как задача таблиц дифференциальной диагностики -- быстрое определение степени уже выявленного заболевания. Можно сказать, что программа DiagnosisPro решала более общую комплексную задачу дифференциальной диагностики. 

Но у этой программы также есть свои недостатки. Первый, и самый основной, это то, что в 2016 году программа перестала поддерживаться и теперь приобрести её у компании-разработчика невозможно. Вторым недостатком было то, что хотя программа и была бесплатной и свободно распространяемой в медицинской сфере, она поддерживала только английский, испанский, французский, арабский и китайский языки. То есть при работе этой программы в России врачи, не владеющие на достаточном уровне английским языком, не смогли бы пользоваться ею. И даже у врачей, владеющих английским языком, возникли бы проблемы при переводе результатов или симптомов из программы на русский язык для внесения в медицинскую карту пациента. Не говоря уже о том, что это существенно замедлило бы работу врачей.

Существует также ещё одна подобная программа -- VisualDx \cite{visualdx}, сайт которой представлен на рисунке~\ref{fig:vdx}. Она также предназначена для решения общей задачи дифференциальной диагностики: врачи могут составлять индивидуальный план пациента, где, отметив его симптомы и основные сведения о человеке, могут просмотреть возможные диагнозы, предложенные программой. Также в VisualDx делает особый упор на визуализацию -- их библиотека содержит более чем 40 000 фотографий различных заболеваний на коже людей разных национальностей, а также врач может просмотреть различные симтомы или болезни в виде рисунка-анимаци.

\begin{figure}
  \includegraphics[scale=0.3]{src/VisDx.png}
  \caption{Сайт программы VisualDx}
  \label{fig:vdx}
\end{figure}

Но у VisualDx также есть свои минусы. Первый -- она, как и  DiagnosisPro не поддержтвает русский язык (только английский, французский, немецкий, китайский, европейский испанский и латиноамериканский испанский языки). Второй -- она платная. Плата вносится ежемесячно или ежегодно, в зависимости от числа компьютеров и пользователей.

И она также, как и DiagnosisPro не реализует электронные таблицы дифференциальной диагностики -- она просто выводит список заболеваний, объясняющих симптомы и различную информацию про эти заболевания. То есть, опять же, решает более глобальную и большую проблему дифференциальной диагностики заболеваний. И таблицы дифференциальной диагностики (или их подобие) присутствуют там лишь в виде текста статьи (рисунок~\ref{fig:vdx_ex_ac}). В обычных случаях это помогает врачу поставить более точный диагноз, глубже разобраться в проблеме, тщательнее изучить тему. Но врачам скорой помощи приходится принимать решения быстро, чаще всего у них нет времени читать статьи и находить среди них нужную информацию. Им надо быстро определить степень заболевания, площадь ожогов, вероятность сердечного приступа у человека и т.п. Именно в таких случаях полезнее информация, содержащая только основные моменты и представленная в виде интерактивных таблиц.
 
\begin{figure}
  \includegraphics[scale=0.3]{src/vdx_ex_article.png}
  \caption{Статья VisualDx о распознавании синяков и ушибов на теле детей}
  \label{fig:vdx_ex_ac}
\end{figure}

\subsection{Обоснование цели и задач, техническое задание на разработку}

В связи со всем вышесказанным моей целью стало разработать интерактивные таблицы дифференциальной диагностики, учтя все минусы и плюсы существующих решений. 

В России существует веб-приложение \url{onmp.ru} \cite{onmp} (рисунок~\ref{fig:onmp0}), созданное специально для заполнения медицинских карт врачами скорой поможи и в нём учтены все спецификации формы медицинской карты вызова. В нём можно заполнять медицинские карты, сохранять их, откладывать в черновики, шаблоны или архив. Оно было разработано сотрудниками и студентами Московского Авиационного Института. Таким образом, можно внедрить таблицы дифференциальной диагностики в это приложение, сдлеать это отдельной его функцией. 

\begin{figure}
  \includegraphics[scale=0.3]{src/onmp0.png}
  \caption{Сайт onmp.ru}
  \label{fig:onmp0}
\end{figure}

Таким образом мы решим две проблемы пердыдущих решений. Во-первых, приложение ОНМП поддерживает русский язык, так как в первую очередь ориентированно именно на Российских врачей, а также в нём настроена форма заполнение такой формы медицинской карты вызова, которую предусматривает законодательство Российской Федерации. Во-вторых, это приложение по стоимости минимум в 8 раз дешевле иностранных аналогов.

Теперь разберём более детально, что надо учесть при внедрении таблиц дифференциальной диагностики.

Чтобы обеспечить врачам доступ к всесторонней информации, надо внедрить большоеколичество таблиц дифференциальной диагностики. Следовательно, для их хранения надо использовать базу данных и разработать метод их хренения в ней.  Этим мы решим проблемы тех сайтов, на которых представлено лишь малое количесвто таблиц дифференциальной диагностики, и которые не справятся с иными случаями.

Чтобы врачи могли быстро найти нужную таблицу, надо реализовать быстрый поиск по названиям этих таблииц. Этим мы решим проблему долгого поиска среди бумажных листов, в интернете и в телеграм-боте.

Чтобы было удобно работать с таблицами, надо реализовать функцию интерактивной работы с ними -- например, дать возможность пользователю выбирать симптомы в строках или баллы в столбцах.

Чтобы снизить психологическую нагрзку на врачей, также надо реализовать автоматический рассчёт результата. Это также снизит вероятность ошибки врачом.

И, наконец, так как приложение ОНМП создано специально для заполнения медицинских карт пациентов, то надо реализовать функцию автоматического внесения результата в медицинскую карту пациента (а именно в поле Status localis). Это также позволит врачам немного быстрее заполнять медицинские карты.

Итак, цель работы: внедрение медицинских таблиц дифференциальной диагностики в веб-приложение ОНМП.

Она состоит из следующих задач:

\begin{enumerate}
  \item Внедрить большое количество таблиц дифференциальной диагностики, используя базу данных;
  \item Реализовать быстрый поиск по таблицам;
  \item Реализовать интерактивную работу с таблицами;
  \item Реализовать автоматический подсчёт результата;
  \item Создать функцию автоматической записи результата в медицинскую карту пациента.
\end{enumerate}


