В современном мире медицинская отрасль активно развивается, и с каждым годом повышается необходимость во внедрении информационных технологий для упрощения и ускорения рабочих процессов. Одним из важнейших аспектов работы медицинских учреждений является отделение неотложной медицинской помощи, где обеспечение оперативного доступа к информации о пациентах, их медицинских картах, а также возможность быстро создавать и редактировать карты являются критически важными. 

С учетом этого, разработка программы заполнения медицинской карты отделения неотложной медицинской помощи представляет собой актуальную и значимую задачу. Создание мобильного приложения, которое будет обеспечивать доступ к медицинским картам и справочникам в удобном и быстром формате, способствует повышению качества предоставления медицинской помощи и снижению вероятности ошибок врачей при заполнении карт. Кроме того, реализация функционала автоматической синхронизации данных с сервером и возможность автономной работы с использованием локальной базы данных SQLite увеличивают надежность и доступность системы в условиях, когда стабильное подключение к интернету может быть ограничено.