\subsection{База данных SQLite}
SQLite — это компактная, самодостаточная база данных, которая обеспечивает хранение данных в файле на устройстве. Она является одной из самых популярных встраиваемых баз данных, используемых для локального хранения данных в мобильных приложениях. SQLite обладает небольшим размером, быстрой скоростью работы и простотой использования, что делает ее идеальным выбором для разработки приложений, которым требуется автономное хранение данных \cite{SQLite}. 

Проведем сравнение SQLite с другими базами данных с возможностью автономной работы приложения, а именно, Realm \cite{Realm} и Couchbase Lite \cite{Couchbase}.

Преимущества SQLite:
\begin{itemize}
        \item открытый исходный код и бесплатное использование;
        \item небольшой размер и низкое потребление ресурсов;
        \item простота интеграции и использования;
        \item поддержка SQL и транзакций.
    \end{itemize}

Недостатки SQLite:
\begin{itemize}
        \item ограниченная производительность при интенсивной многопоточной записи;
        \item меньшее количество дополнительных функций по сравнению с альтернативными решениями.
    \end{itemize}


Преимущества Realm:
\begin{itemize}
        \item высокая производительность и оптимизация для мобильных устройств;
        \item объектно-ориентированный подход, что упрощает работу с данными;
        \item поддержка платформы для обмена данными между устройствами.
    \end{itemize}

Недостатки Realm:
\begin{itemize}
        \item меньшая стабильность по сравнению с SQLite;
        \item ограниченная поддержка SQL;
        \item лицензионные ограничения и стоимость использования.
    \end{itemize}

Преимущества Couchbase Lite:
\begin{itemize}
        \item масштабируемость и производительность;
        \item гибкий механизм синхронизации данных между устройствами;
        \item поддержка JSON-документов и NoSQL.
    \end{itemize}

Недостатки  Couchbase Lite:
\begin{itemize}
        \item возможность интеграции и настройки;
        \item больший размер и потребление ресурсов по сравнению с SQLite;
        \item лицензионные ограничения и стоимость использования.
    \end{itemize}
    
Проанализировав плюсы и минусы, было решено использовать базу данных SQLite в разрабатываемом приложении.