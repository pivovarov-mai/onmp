\subsection{REST API}
REST (Representational State Transfer) — это архитектурный стиль проектирования веб-сервисов, призванный обеспечить легкость взаимодействия между клиентами и серверами в сети. Он основан на использовании стандартных протоколов, методов и соглашений, таких как HTTP, URI и JSON или XML для обмена данными \cite{REST}. 

Основные принципы REST API:
\begin{enumerate} 
     \item Stateless: Каждый запрос от клиента к серверу должен содержать всю информацию, необходимую для обработки и выполнения запроса. Сервер не должен хранить информацию о предыдущих запросах клиента между запросами.

    \item Cacheable: Ответы от сервера могут быть кэшированы на стороне клиента, если сервер указывает, что это возможно. Кэширование может снизить нагрузку на сервер и улучшить производительность приложения.

    \item Client-Server: Архитектура REST разделяет клиентские и серверные обязанности, что обеспечивает гибкость и возможность развития каждой части независимо друг от друга.

    \item Layered System: REST позволяет разделять архитектуру на слои, что облегчает управление и разработку каждого слоя в отдельности, а также увеличивает безопасность системы.

    \item Uniform Interface: Использование единого интерфейса между клиентами и серверами облегчает разработку и взаимодействие между различными компонентами системы.
\end{enumerate}

Работа с REST API обычно включает использование стандартных HTTP-методов, таких как GET, POST, PUT, PATCH и DELETE, для выполнения операций с ресурсами на сервере. Ресурсы идентифицируются через URI (Uniform Resource Identifier), а сами данные передаются в формате JSON или XML.

Преимущества использования REST API:
\begin{itemize}
    \item Простота и гибкость: REST API легко понять и использовать, так как оно базируется на стандартных протоколах и методах. Кроме того, REST API может быть легко расширено и модифицировано для поддержки новых функций.

    \item Масштабируемость: Благодаря разделению обязанностей между клиентами и серверами, а также использованию кэширования и многослойной архитектуры, REST API позволяет создавать масштабируемые приложения, которые могут обслуживать большое количество пользователей.
    \item Независимость от платформы и языка: REST API может быть использовано практически с любым языком программирования и на любой платформе, поскольку оно использует стандартные протоколы и методы для обмена данными. Это делает REST API универсальным и доступным для различных приложений и платформ.

    \item Широкая поддержка и интеграция: Большинство современных веб-сервисов и приложений поддерживают REST API, что позволяет легко интегрировать различные сервисы и функции в одно приложение.

    \item Безопасность: Использование многослойной архитектуры и стандартных протоколов позволяет обеспечить безопасность данных и взаимодействия между клиентами и серверами.
\end{itemize}

Однако стоит отметить и некоторые недостатки REST API:

\begin{itemize}
    \item Необходимость версионирования: Изменения в API могут привести к несовместимости с предыдущими версиями, что требует внедрения версионирования и поддержки старых версий API.

    \item Производительность: В некоторых случаях использование текстовых форматов (JSON, XML) для обмена данными может снизить производительность приложения, особенно при работе с большим объемом данных или высокой нагрузкой.
\end{itemize}

Тем не менее, REST API является одним из наиболее популярных и широко используемых стилей проектирования веб-сервисов и приложений и в данной выпускной работе REST API используется для обеспечения взаимодействия между клиентским приложением и сервером, что позволяет реализовать регистрацию, авторизацию и синхронизацию медицинских карт.