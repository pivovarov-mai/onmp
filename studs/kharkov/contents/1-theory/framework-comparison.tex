
\subsection{Сравнение фреймворков для разработки мобильных приложений}
Для начала проведем сравнительный анализ популярных фреймворков для разработки мобильных приложений, а именно .NET MAUI, React Native и Flutter. Анализ проведем с точки зрения основных характеристик, таких как язык программирования, экосистема, производительность и кросс-платформенность. 

.NET MAUI (Multi-platform App UI), выпущенный в 2021 году, является продолжением Xamarin.Forms и представляет собой фреймворк для разработки кросс-платформенных приложений с использованием C\# и .NET. MAUI позволяет создавать единую кодовую базу для разработки приложений под разные платформы, включая Android, iOS, macOS и Windows \cite{MAUI}. 

React Native - это фреймворк, разработанный в 2015 году, для создания кросс-платформенных мобильных приложений на языке JavaScript с использованием React. React Native позволяет использовать единую кодовую базу для разработки приложений под Android и iOS \cite{React}. 

Flutter - фреймворк, разработанный Google в 2018 году, для создания кросс-платформенных приложений с использованием языка программирования Dart. Flutter позволяет разрабатывать приложения для Android, iOS, web и десктопных платформ \cite{Flutter}. 

Сравнение основных характеристик:
\begin{enumerate} 
    \item Производительность:
    \begin{itemize}
        \item MAUI: Близка к нативной за счет компиляции в промежуточный язык (IL) и использования рантайма Mono/.NET.
        \item React Native: Ниже нативной из-за JavaScript-моста для взаимодействия с нативными компонентами.
        \item Flutter: Высокая производительность благодаря компиляции в нативный код и использованию собственного движка отрисовки.
    \end{itemize}
    
    \item Кросс-платформенность:
    \begin{itemize}
        \item MAUI: Поддерживает Android, iOS, macOS и Windows.
        \item React Native: Основная поддержка для Android и iOS, с ограниченной поддержкой для Windows и macOS через сторонние решения.
        \item Flutter: Поддерживает Android, iOS, web, а также десктопные платформы, включая Windows, macOS и Linux.
    \end{itemize}

    \item Экосистема и поддержка:
    \begin{itemize}
        \item MAUI: Хорошая интеграция с экосистемой .NET, которая имеет большое сообщество, и поддержка Microsoft.
        \item React Native: Большое сообщество, множество библиотек и хорошая поддержка разработчиков.
        \item Flutter: Активно развивается Google, имеет растущее сообщество и набор готовых компонентов.
    \end{itemize}
\end{enumerate}

В результате сравнения можно сделать вывод, что выбор фреймворка для разработки мобильных приложений зависит от предпочтений разработчиков и предполагаемой поддержки платформ. Так как .NET MAUI обеспечивает достаточно высокую производительность и поддержку множества платформ, мной было принято использовать его для разработки приложения.