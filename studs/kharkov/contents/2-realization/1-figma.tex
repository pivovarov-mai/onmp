\subsection{Создание макета интерфейса в Figma}

Перед началом работы необходимо создать макет интерфейса приложения, который удовлетворял бы запросам пользователей, было принято решение использовать Figma. Figma является мощным онлайн-инструментом для создания дизайна пользовательского интерфейса, который позволяет разработчикам и дизайнерам работать над проектом совместно, делиться макетами и получать обратную связь. Она предоставляет широкий набор инструментов для создания прототипов, редактирования графики и визуализации пользовательского интерфейса. Перед началом разработки интерфейса программы в Figma, были проведены исследования и анализ требований пользователей. 

Процесс разработки интерфейса в Figma начался с создания основного макета, включающего в себя различные элементы пользовательского интерфейса, такие как навигационные панели, кнопки, поля ввода и т. д. Эти элементы были размещены на главном экране программы с учетом удобства использования и логической структуры. Затем были разработаны отдельные экраны и вкладки, соответствующие функциональности программы. Были учтены требования к расположению элементов интерфейса, цветовым схемам, шрифтам и прочим аспектам дизайна. В процессе работы было обращено внимание на создание интуитивно понятного в использовании интерфейса, который бы удовлетворял потребности пользователя и способствовал эффективному взаимодействию с программой.

Figma также предоставляет возможность создания интерактивных прототипов, которые позволяют смоделировать взаимодействие пользователя с интерфейсом программы. В процессе разработки интерфейса были созданы прототипы, которые помогли проверить функциональность и удобство использования интерфейса. Были проведены тестирования прототипов с участием пользователей, что позволило выявить возможные проблемы и внести необходимые корректировки.

В результате  был разработан детальный и проработанный интерфейс программы, соответствующий требованиям и ожиданиям пользователей. Интерфейс был визуализирован с использованием различных графических элементов и цветовых схем, что придало программе эстетически приятный и современный вид. Ссылку на макет можно посмотреть на рисунке ~\ref{fig:qrcode_figma}.