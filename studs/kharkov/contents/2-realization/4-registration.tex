\subsection{Разработка страниц авторизации и регистрации}
После создания базы данных были разработаны страницы авторизации и регистрации, которые является важными элементами в обеспечении безопасности приложения и доступа к персональным данным пользователей. На странице авторизации представлены два поля для ввода: почта и пароль, что позволяет пользователям войти в систему с использованием своих учетных данных. Страница регистрации содержит поля для ввода почты, два поля для пароля (подтверждение пароля), имя и фамилию, что дает возможность новым пользователям создать аккаунт и начать использовать приложение. Также на странице авторизации имеется опция сохранения пароля, что упрощает процесс повторного входа в приложение. Если пользователь выбирает эту опцию, то при следующем запуске приложения вход в аккаунт будет выполнен автоматически. Это существенно экономит время пользователя и повышает удобство использования приложения.

Для обеспечения корректной работы приложения и предотвращения отправки некорректных данных на сервер, была реализована валидация введенной информации: c помощью регулярного выражения проверяется корректность введенной почты, сравниваются пароли и проверяется, что поля не пустые - это позволяет уменьшить нагрузку на сервер и ускорить процесс авторизации и регистрации.

Для отправки данных на сервер используется технология REST API библиотека HttpClient: к примеру, на этапе авторизации формируется json файл из почты и пароля и отправляется на определенный интернет адрес, а в качестве ответа сервер отправляет ошибку, если какие-то данные некорректны или токен доступа для последующего общения с сервером. В будущем планируется использовать дополнительное шифрование для еще большей безопасности при передаче данных.

В процессе разработки особое внимание было уделено обработке ошибок и уведомлениям пользователей. Если ошибка выявлена на первом этапе (до отправки на сервер), информация об ошибке будет выведена в специальном поле на странице. Если ошибка возникает после отправки данных на сервер, пользователю будет показано всплывающее уведомление об ошибке, что позволяет оперативно реагировать на проблемы и предоставлять пользователям актуальную информацию о состоянии системы. Для повышения удобства использования приложения на странице авторизации были реализованы различные визуальные элементы и анимации. При неверном формате почты текст поля выделяется красным, что позволяет пользователям быстро заметить ошибку и исправить ее, а если пароли не совпадают при регистрации, то поля для их ввода немного \enquote{дрожат}, что указывает на некорректность введенных данных и направляет внимание пользователя на этот момент.

В случае отсутствия доступа к интернету, приложение обеспечивает возможность автоматического входа в аккаунт без необходимости повторной авторизации, если пользователь уже входил в систему ранее. Эта функция позволяет пользователям продолжать работать с локальными данными, даже если связь с сервером временно недоступна.