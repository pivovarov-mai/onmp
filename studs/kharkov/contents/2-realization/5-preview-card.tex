\subsection{Разработка страницы поиска медицинских карт}
В данном подразделе рассмотрим процесс разработки страницы поиска медицинских карт.

Основные элементы пользовательского интерфейса страницы включают поле поиска, фильтры, список с результатами поиска и кнопку для создания новой карточки. Фильтры позволяют пользователю выбирать тип карты (архив, шаблон, готовая, черновик) и дату создания, для получения более точных результатов поиска. Результаты запроса отображаются в списке, где пользователь может просмотреть краткую информацию о карте, такую как название, тип и дату создания. 

Работа поиска происходит следующим образом: после изменения поля поиска или изменения типа выводимых карты, вызывается функция, в которой происходит обращение к базе данных. В этом запросе учитываются все фильтры, выставленные пользователем, а в результате возвращается определенное количество медицинских карт, которые соответствуют фильтрам. Для оптимальной работы приложения и снижения нагрузки на устройство, в списке было решено использовать бесконечную прокрутку, которая позволяет загружать элементы медицинских карт по 30 штук при достижении конца списка. Такой подход обеспечивает более плавную загрузку данных и улучшает пользовательский опыт.

При взаимодействии с медицинскими картами пользователь может выполнить свайп вправо для перемещения элемента в архив. Свайп влево приводит к полному удалению элемента. Для более подробного просмотра и редактирования карты, пользователь может просто нажать на нее, после чего открывается соответствующая страница. В нижней части страницы поиска расположена кнопка создания карточки, с помощью которой можно создать новую медицинскую карту.

