\subsection{Разработка страницы справочника}
В процессе разработки программы важной составляющей являлась страница справочника. Этот элемент обеспечивает необходимую функциональность для взаимодействия с терминами и данными, которые используются врачами в ходе ведения и обработки медицинских карт. Внедрение и разработка страницы справочника представляло собой набор ключевых шагов, включающих в себя разработку интерфейса пользователя, интеграцию функциональности поиска и синхронизацию с базой данных и сервером.

Поиск данных в справочнике был реализован с использованием двухфазной стратегии. Изначально поиск происходит в рамках локальной базы данных, что обеспечивает мгновенный ответ на запрос пользователя и, тем самым, сокращает нагрузку на сервер. Однако, если искомые данные отсутствуют в локальном хранилище, система автоматически переходит к второй фазе - обращению к серверу через API. Этот процесс предполагает выполнение запроса к серверу и получение необходимых данных. Важным аспектом здесь является кэширование данных. Полученная с сервера информация сохраняется в локальной базе данных, что позволяет быстро получать доступ к ней при повторных запросах. Этот подход обеспечивает высокую скорость обработки данных и уменьшает зависимость от скорости и стабильности интернет-соединения.

В случае, когда пользователь выбирает определенный элемент справочника для просмотра дополнительной информации, система работает абсолютно также. Если данные уже находятся в локальной базе, они мгновенно выводятся на экран. Если же данные располагаются на сервере, система сперва добавляет эту информацию в локальное хранилище, а затем отображает ее пользователю.

Таким образом, разработка страницы справочника была сложным и многоэтапным процессом. Включая создание функционального и удобного интерфейса, интеграцию двухфазного поиска данных и работу с базой данных и сервером, мы создали мощный инструмент для поиска и отображения данных справочника, который может быть использован медицинскими работниками в процессе их работы с приложением.