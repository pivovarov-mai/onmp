\subsection{Тестирование программы}
В данной работе для тестирования различных функциональных аспектов разработанной программы использовался фреймворк xUnit. xUnit является одной из наиболее популярных библиотек для проведения модульного тестирования в среде .NET, с его помощью мы сможем автоматизировать большую часть процесса тестирования и убедиться, что каждый из компонентов программы работает так, как ожидается.

Особенное внимание при тестировании было уделено процессам регистрации и авторизации. Это особо критичные функции, так как они обеспечивают безопасность и контроль доступа к медицинским картам. Тестировались различные сценарии, включая попытки регистрации с уже существующим именем пользователя, ввод некорректного пароля и т.д.

Пример функций авторизации пользователя на рисунке~\ref{src:tests}.

\begin{figure}
\lstinputlisting[language=C++]{inc/tests.cs}
\caption{Функции тестирования авторизации}
\label{src:tests}
\end{figure}

Следующим аспектом, который был подвергнут тестированию, стали функции поиска. Разработанная система включает в себя возможность поиска как среди медицинских карт, так и в справочнике. Было важно убедиться, что результаты поиска являются точными и релевантными, а также корректно обрабатываются случаи, когда поиск не дает результатов.

Также мы тестировали работу с медицинскими картами, включая их создание и редактирование. Здесь мы проверяли не только корректность хранения и обработки данных, но и наличие всей необходимой информации на картах, а также удобство и интуитивность процесса их редактирования.

Особое внимание было уделено тестированию корректности работы генерации вопросов с использованием шаблона Компоновщика. Это важный элемент нашей системы, который обеспечивает гибкость и возможность адаптации под различные сценарии использования.

В результате тестирования мы убедились, что разработанная система работает стабильно, корректно и отвечает всем предъявляемым к ней требованиям. Это позволяет нам быть уверенными в ее готовности к внедрению и использованию в реальных условиях.