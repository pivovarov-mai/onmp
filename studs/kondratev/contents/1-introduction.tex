\introduction % Структурный элемент: ВВЕДЕНИЕ
Медицина, как отрасль, должна постоянно адаптироваться и развиваться вместе с научными достижениями. Однако некоторые области до сих пор полагаются на бумажную документацию, что снижает эффективность работы сотрудников и потенциально увеличивает количество ошибок при обработке информации. Для решения этой проблемы в качестве темы выпускной квалификационной работы бакалавра было выбрано создание функционала для расшифровки и анализа на острые патологии снимка ЭКГ для приложения ONMP. Основной целью данного проекта является повышение эффективности работы медицинских работников и снижение риска неблагоприятных исходов для пациентов.

Приложение ONMP призвано обеспечить медицинскому персоналу быстрый доступ к необходимой информации, упростить процессы документирования, минимизировать бумажную работу и повысить общую эффективность.

Целями приложения можно выделить: 
\begin{itemize}
    \item расшифровка и анализ снимка ЭКГ на острые патологии,
    \item поиск по энциклопедии с сортировкой,
    \item возможность обрабатывать шаблоны больного,
    \item быстро получать личную информацию больного,
    \item генерировать pdf-файл по шаблону для печати,
    \item иметь доступ к функционалу без интернета.
\end{itemize}

В настоящее время разработка подобных приложений находится на ранней стадии, имеется лишь прототип с ограниченной функциональностью. Тем не менее, очевидно, что эта тема актуальна и имеет огромный потенциал для дальнейшего развития. Такие приложения способны значительно улучшить жизнь людей, помочь им более эффективно распоряжаться своим временем и силами, а также повысить общую эффективность медицинских услуг.

Об актуальности именно моей части разработки.
Наиболее распространенной причиной смерти является ишемическая болезнь сердца, на которую приходится 16\% от общего числа смертей в мире. Ранняя и точная диагностика сердечно-сосудистых заболеваний имеет большое значение для снижения уровня смертности и улучшения результатов лечения пациентов. Электрокардиограмма (ЭКГ) - это неинвазивный и широко используемый диагностический инструмент, который регистрирует электрическую активность сердца. Она предоставляет важнейшую информацию о ритме, частоте и проводящей системе сердца и может помочь выявить широкий спектр сердечных заболеваний, включая острые патологии.

Однако интерпретация ЭКГ требует специальной подготовки и опыта, что может отнимать много времени и чревато ошибками. Программная расшифровка ЭКГ, с другой стороны, использует передовые алгоритмы и компьютерное программное обеспечение для анализа и интерпретации сигналов ЭКГ, предлагая ряд преимуществ по сравнению с ручной расшифровкой. Среди них - исключение возможности человеческой ошибки при чтении и интерпретации ЭКГ-сигнала и обеспечение быстрой и точной диагностики острых патологий.

Важность программной расшифровки ЭКГ в кардиологии еще больше подчеркивается растущим спросом на услуги удаленного здравоохранения. С появлением телемедицины пациенты могут получать медицинские консультации и диагнозы из удаленных мест. Программное обеспечение для расшифровки ЭКГ может сыграть решающую роль в облегчении этого процесса, обеспечивая точную и надежную интерпретацию сигналов ЭКГ.

Кроме того, использование цифровых технологий в здравоохранении становится все более важным и необходимым для врачей. Цифровая медицинская карта обеспечивает быстрый и удобный доступ к информации о пациенте, повышая эффективность и точность диагностики и лечения. Она также позволяет хранить и обрабатывать большие объемы данных о пациентах, что необходимо для проведения исследований и разработки новых методов лечения. Кроме того, она предоставляет врачам доступ к общей истории лечения пациента, включая предыдущие медицинские состояния, что повышает качество и эффективность лечения.

Использование цифровых медицинских карт также снижает вероятность ошибок и искажений при заполнении и хранении информации о пациенте, обеспечивая безопасность и конфиденциальность медицинских данных. Кроме того, это современный и удобный подход к организации работы медицинских учреждений и обслуживанию пациентов, способствующий повышению уровня медицинского обслуживания и общего качества жизни населения.

В заключение следует отметить, что разработка точного и надежного программного обеспечения для диагностики и выявления острых патологий на изображениях ЭКГ имеет большое значение в области кардиологии. Предложенное решение может помочь повысить качество и эффективность медицинских услуг, особенно в экстренных ситуациях. Кроме того, использование цифровых технологий в здравоохранении становится все более важным и необходимым для врачей, обеспечивая современный и удобный подход к организации работы медицинских учреждений и обслуживанию пациентов. 

Таким образом, выполненная работа актуальна с научной, методологической, теоретической и практической точек зрения, способствуя повышению качества медицинских услуг и улучшению результатов лечения пациентов.

