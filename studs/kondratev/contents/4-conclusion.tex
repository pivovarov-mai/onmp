\section{Заключение}
В результате обширных исследований и разработок было создано комплексное решение, которое решает ключевые задачи и проблемы рассматриваемого проекта. Это решение представляет собой значительный прогресс в данной области, особенно в сфере анализа и интерпретации электрокардиограмм (ЭКГ).

Основной функцией системы является нормализация фотографий ЭКГ - процесс, включающий преобразование аналоговых сигналов, запечатленных на снимках, в оцифрованные представления. Этот важнейший шаг позволяет проводить дальнейший анализ и манипуляции с данными ЭКГ, что приводит к более глубокому пониманию сердечной деятельности. Система отлично справляется с этой задачей, обеспечивая точное и надежное преобразование фотографий ЭКГ в оцифрованные сигналы.

После оцифровки сигналов ЭКГ система использует передовые алгоритмы для извлечения основных характеристик этих сигналов. Сюда входят точные измерения продолжительности, амплитуды и морфологии, которые являются основополагающими для оценки общего состояния здоровья и функциональности сердца. Получая аналитические показания на основе этих характеристик, система предоставляет медицинским работникам ценную информацию для диагностики и планирования лечения.

Работа системы была тщательно оценена, и она продемонстрировала высокий уровень точности и надежности при обнаружении отклонений в сигналах ЭКГ. Сравнивая извлеченные характеристики с установленными нормальными диапазонами, система эффективно выявляет отклонения от ожидаемых закономерностей, что позволяет обнаружить заболевания, связанные с сердцем, на ранней стадии. Такое раннее обнаружение имеет решающее значение для предотвращения дальнейших осложнений и своевременного вмешательства.

При разработке системы были тщательно учтены требования и потребности пользователей. В результате был разработан удобный интерфейс, позволяющий медицинским работникам легко вводить снимки ЭКГ и получать четкие и ясные аналитические данные. Кроме того, система демонстрирует универсальность, поддерживая широкий спектр форматов изображений ЭКГ, обеспечивая совместимость с различными источниками данных, часто встречающимися в клинических условиях.

В целом, достижение целей проекта демонстрирует способность системы нормализовать и оцифровывать фотографии ЭКГ, извлекать основные характеристики и предоставлять аналитические показания. Система способна оказать значительную помощь медицинским работникам в диагностике заболеваний, связанных с сердцем, что позволит принимать своевременные и точные решения о лечении. Дальнейшая проверка и тестирование будут способствовать совершенствованию и улучшению системы, что в конечном итоге принесет пользу медицинскому сообществу и улучшит обслуживание пациентов.