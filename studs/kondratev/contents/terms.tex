\newglossaryentry{id5}{
    name={Зубцы},
    description={представляют собой последовательность деполяризации и реполяризации предсердий и желудочков} 
}
\newglossaryentry{id6}{
    name={Интервал},
    description={это время между двумя конкретными событиями ЭКГ.} 
}
\newglossaryentry{id8}{
    name={Комплекс},
    description={сочетание нескольких сгруппированных волн. Единственный основной комплекс на ЭКГ — это комплекс QRS.} 
}
\newglossaryentry{id1}{ % Нужны разные id, можно ставить просто последовательно
    name={Отделение неотложной медицинской помощи},
    description={это отделение в составе более крупного медицинского учреждения, такого как поликлиника, амбулатория или центр общей врачебной практики. Его основная цель - оперативно и эффективно удовлетворять неотложные потребности пациентов в медицинской помощи, обеспечивая при этом их общее благополучие и безопасность. Неотложная медицинская помощь в таких условиях может оказываться парамедиками в качестве первичной доврачебной помощи или врачами, работающими в амбулатории и оказывающими первичную медицинскую помощь.} 
}

\newglossaryentry{id3}{
    name={Программное обеспечение},
    description={это совокупность программных и документационных инструментов, предназначенных для разработки и управления системами обработки данных с использованием вычислительной техники.} 
}
\newglossaryentry{id7}{
    name={Сегмент},
    description={длина между двумя конкретными точками на ЭКГ, которые должны иметь базовую амплитуду.} 
}
\newglossaryentry{id4}{
    name={Фреймворк},
    description={(англ. framework — «остов, каркас, структура») — готовая модель в IT, заготовка, шаблон для программной платформы, на основе которого можно дописать собственный код. Проще говоря, фреймворк предлагает промежуточный вариант с точки зрения гибкости и сложности, находящийся между разработкой кода с нуля и использованием системы управления контентом. Если создание кода с нуля похоже на свободное рисование на чистом листе бумаги любыми инструментами, а использование системы управления контентом - на раскрашивание в заранее определенных границах, то использование фреймворка сравнимо с рисованием в блокноте в определенных рамках. Он по-прежнему позволяет свободно рисовать все, что угодно, но дает указания и предопределенные границы, облегчая процесс.
    Фреймворки выгодны тем, что они решают множество сложных деталей, таких как управление файловой системой и базой данных, обработка ошибок и безопасность программы. Они берут на себя все эти аспекты, освобождая разработчиков от необходимости изобретать колесо для каждого проекта.
    Путаница между фреймворками и библиотеками возникает из-за их схожего функционала. В то время как фреймворк - это приложение, используемое для создания веб-сайтов или приложений, библиотека - это предварительно созданный компонент, предназначенный для выполнения определенных задач в рамках проекта. Например, существуют библиотеки для обработки файлов и вывода изображений на экран.} 
}
\newglossaryentry{id2}{
    name={Электрокардиограмма},
    description={ценный и неинвазивный метод, используемый для оценки работы сердца. Она дает кардиологам важные и уникальные сведения о состоянии сердца пациента, не требуя при этом никакой специальной подготовки. Информация, полученная с помощью ЭКГ, является подробной и не может быть легко заменена, что делает ее незаменимым инструментом в оценке здоровья сердца.} 
}