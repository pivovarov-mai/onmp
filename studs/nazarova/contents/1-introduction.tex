\introduction % Структурный элемент: ВВЕДЕНИЕ

В повседневной жизни миллионы людей по всему миру пользуются веб-приложениями. Это те приложения, которые имеют ряд преимуществ перед традиционными приложениями: их не надо устанавливать на телефон, так как они доступны через интернет и работают в браузере на компьютере или мобильном устройстве пользователя. Это упрощает использование и обновление приложения, так как изменения в программном обеспечении могут быть внесены на сервере, а не на каждом устройстве пользователя. Они могут предоставлять различные функции, такие как социальные сети, онлайн-магазины, банковские системы, онлайн-игры, приложения для обработки фотографий и многое другое.


Веб-приложения также обеспечивают легкую интеграцию с другими системами и приложениями, что позволяет пользователям управлять различными задачами и процессами в одном месте.
Интернет-приложения состоят из двух основных компонентов: frontend и backend. Frontend-часть веб-приложения представляет собой пользовательский интерфейс, который взаимодействует с пользователем на стороне клиента, то есть в браузере. Разработчики отвечают за создание самого пользовательского интерфейса, его стилизацию и добавление разных динамических элементов.  Backend - часть представляет собой серверную часть, которая отвечает за обработку запросов пользователя, управление базами данных и бизнес-логику. 


Веб-приложения работают следующим образом: пользователь взаимодействует с frontend-частью, запрашивая отдельные страницы или функциональности, затем frontend-часть отправляет запросы на получение каких-либо данных на backend-часть, которая обрабатывает эти запросы, выполняет необходимые действия и отправляет ответ обратно на frontend-часть, которая затем отображает результат пользователю.

В данной работе осуществляется внедрение алгоритмов оказания медицинской помощи в веб-приложение OНМП.  Это приложение предназначено для упрощения оказания медицинской помощи пациентам. Оно позволяет врачам скорой помощи быстро заполнять карты вызовов, содержащие основную информацию о пациенте, его состоянии и мероприятиях, необходимых для оказания медицинской помощи.
В приложении, помимо алгоритмов, которые помогают врачам скорой помощи быстро определить необходимые мероприятия для оказания помощи, представлены: таблицы дифференциальной диагностики, которые помогают врачам сравнить симптомы и выбрать правильный диагноз; калькулятор дозировок, который помогает врачам точно определить необходимую дозу лекарственных средств для пациентов;справочник, в котором содержится информация, которая может пригодиться при оказании помощи.


Актуальность темы данной работы связана с улучшение качества медицинской помощи и сокращение времени оказания помощи являются ключевыми приоритетами в системе здравоохранения. Использование электронных алгоритмов и интерфейса за контролем их выполнения может помочь предотвратить медицинские ошибки, которые могут привести к серьезным последствиям для пациентов. Поэтому разработка и внедрение таких систем являются крайне важными для улучшения качества медицинской помощи и повышения безопасности пациентов.

Таким образом, выполненная работа актуальна и с научно-методической/теоретической, и с практической точек зрения.


Цель работы – разработать понятный пользовательский интерфейс по алгоритмам оказания медицинской помощи и внедрить его в web-приложение.


Для достижения поставленной цели в работе были решены следующие задачи:
\begin{itemize}
    \item проведение анализа существующих приложений, в которых есть функционал, отвечающий за предосталвние информациии об алгоритмах медицинской помощи;
    \item выявление плюсов и минусов существующих решений, создание макета итогового решения;
    \item выявление данных, которые должны храниться в базе данных, приведение их к одной структуре;
    \item изучение существующих технологий для реализации необходимого интерфейса, выбор наиболее подходящей;
    \item изучение языка программирования JS, выбранного фреймворка React и библиотеки компонентов MUI;
    \item реализована страница с алгоритмами по коду МКБ;
    \item реализован поиск по названию кода МКБ;
    \item настроена возможность отмечать врачом оказанную помощь;
    \item реализована связь с удаленным сервером и возможностью получать оттуда необходимые данные.
\end{itemize}


Работа основывалась на следующих инструментах и методах: 

\begin{itemize}
    \item JavaScript. Это язык программирования, который используется для создания интерактивных элементов и логики веб-приложения. JavaScript позволяет обрабатывать события, взаимодействовать с пользователем и выполнять различные операции на клиентской стороне;
    \item React. Это JavaScript-библиотека, которая позволяет разрабатывать пользовательский интерфейс веб-приложений. React обеспечивает эффективное управление состоянием приложения, модульность компонентов и возможность повторного использования кода;
    \item MUI. Это библиотека компонентов для React, которая предоставляет готовые стилизованные элементы интерфейса. MUI содержит широкий набор компонентов, таких как кнопки, формы, таблицы и другие, которые могут быть легко интегрированы в приложение и обеспечивают единообразный и современный дизайн;
    \item Git. Это распределенная система управления версиями, которая используется для контроля и управления изменениями в коде приложения. Git обеспечивает возможность совместной работы над проектом и отслеживания изменений в исходном коде.
\end{itemize}

Основными результатами, полученными в работе, являются:

\begin{itemize}
    \item разработка web-страницы алгоритмов оказания медицинской помощи для web-приложения. В результате работы была разработана страница для организации неотложной медицинской помощи с использованием технологий React и MUI. Web-страница включает в себя функционал поисковой строки, аккордеонов и checkbox-элементов, что обеспечивает удобный и интуитивно понятный интерфейс для пользователей;
    \item улучшение процесса оказания медицинской помощи. Разработанный функционал позволяет оптимизировать процессы работы медицинского персонала. Врачи могут быстро находить необходимую информацию, использовать поисковую функцию для быстрого доступа к алгоритмам. Это способствует повышению качества и эффективности медицинской помощи;
    \item улучшение доступности и удобства использования. Использование современных технологий и интерфейсных компонентов, таких как MUI, позволяет создать приятный пользовательский интерфейс. Пользователи могут легко взаимодействовать с приложением, выполнять поиск, развернуть аккордеоны и выбрать необходимые опции с помощью checkbox-элементов.
\end{itemize}


Результаты данной работы предназначены для интеграции разработанного функционала в существующую систему оказания медицинской помощи. Они представляют собой готовое решение, которое может быть интегрировано в существующую инфраструктуру и использовано медицинским персоналом для улучшения процесса оказания помощи пациентам.


Использование разработки позволяет современному медицинскому персоналу эффективно использовать технологии информатики для улучшения процесса оказания медицинской помощи, повышения доступности информации и улучшения координации работы команды. Это приводит к более качественной и эффективной медицинской помощи, что в свою очередь благоприятно сказывается на здоровье и благополучии пациентов.


