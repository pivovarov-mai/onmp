\section{ТЕСТИРОВАНИЕ}

Тестирование алгоритмов играет важную роль в разработке программного обеспечения и помогает убедиться в их корректности, надежности и эффективности.

Основные цели тестирования алгоритмов оказания медицинской помощи включают:

\begin{itemize}
    \item верификация. Проверка алгоритмов на соответствие заданным требованиям и спецификациям. Это включает проверку правильности логики, правильности выполнения каждого шага и соответствие ожидаемым результатам;
    \item интеграция. Проверка правильной работы алгоритмов в контексте веб-приложения и его взаимодействия с другими компонентами и системами. Это включает проверку, что алгоритмы правильно интегрированы и работают внутри приложения;
    \item проверка правильности результатов. Провести тестирование, чтобы убедиться, что правильный диагноз и связанная информация отображаются при вводе соответствующего запроса;
    \item тестирование checkbox. Провести тестирование, чтобы убедиться, что при переходе на другую страницу выскакивает сообщение, предупреждающее пользователя о том, что не все пункты по оказанию помощи выбраны(если таковые есть), убедиться в том, что выбранные врачом пукнты по оказанию помощи переносятся в карту вызова.
\end{itemize}

Результаты тестирования подтверждают соответствие разработанного функционала заявленным требованиям и ожиданиям. В ходе тестирования было проверено несколько аспектов функциональности и производительности приложения:
\begin{itemize}
    \item правильность результатов. При проведении тестов по поиску диагнозов по коду МКБ было установлено, что приложение верно определяет соответствующий диагноз и предоставляет связанную информацию, тактику, рекомендации и медицинскую помощь. Результаты соответствуют ожиданиям и согласуются с предоставленными данными;
    \item обработка checkbox. При тестировании функционала checkbox, было проверено, что при не выбранных checkbox в поддиагнозе в подразделе оказанной медицинской помощи при попытке перехода в другой раздел, правильно выдается сообщение о том, что не вся оказанная помощь была выбрана. Также выбранные пункты сохраняются в карту.
\end{itemize}

