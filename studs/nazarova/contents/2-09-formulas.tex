\section{ФОРМУЛЫ}

Можно сделать заинлайненую формулу: $E = mc^2$. Можно сделать формулу по центру (окружение \texttt{equation}):

\begin{equation}
    \label{f1}
    E = mc^2,
\end{equation}

где $E$ --- энергия,\par $m$ --- масса,\par $c$ --- скорость света.

В таком случае точки и знаки препинания лучше оставить внутри формулы. Обратите внимание, что пояснения указываются в порядке появления в формуле, каждое пояснение с новой строки (для этого используйте \texttt{\textbackslash par} после каждого пояснения). \textbf{Необходимо} оставлять пустую строку до (и после) окружений \texttt{equation}, \texttt{align}, \texttt{gather}, \texttt{split}, etc. иначе не будет нужного пустого расстояния.  

Также можно ссылаться на формулу \ref{f1}.

Несколько формул в окружении \texttt{align}:

\begin{align}
x'_1 &= \frac{-3 x_1 + 4 x_2 - x_3}{2 \tau} \\
x'_i &= \frac{x_{i+1} - x_{i-1}}{2 \tau}, \qquad i = \overline{2, n-1} \\
x'_n &= \frac{x_{n-2} - 4 x_{n-1} + 3 x_n}{2 \tau}
\end{align}

Несколько формул в окружении \texttt{gather}:

\begin{gather} 
2x - 5y =  8 \\ 
3x^2 + 9y =  3a + c
\end{gather}

Формула на нескольких строках в окружении \texttt{split}. По умолчанию здесь тоже будет большое расстояние между строками. Чтобы этого избежать используйте команду \texttt{\textbackslash setlength\{\textbackslash jot\}\{3pt\}} внутри \texttt{equation}:

\begin{equation}
\setlength{\jot}{3pt}
\begin{split}
F = {} &\{F_{x} \in  F_{c} : (|S| > |C|) \cap {} \\
& \cap (\mathrm{minPixels}  < |S| < \mathrm{maxPixels}) \cap {} \\
& \cap (|S_{\mathrm{conected}}| > |S| - \varepsilon) \}
\end{split}
\end{equation}