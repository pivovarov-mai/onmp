\conclusion

В заключение, данная выпускная квалификационная работа бакалавра исследовала тему <<Внедрение и контроль выполнения алгоритмов оказания медицинской помощи в web-приложение onmp.ru>>. В процессе исследования были рассмотрены различные аспекты использования web-приложений в области оказания медицинской помощи и их влияние на процесс оказания помощи.

Анализируя текущую ситуацию в медицинском обслуживании и роль технологий информатики, можно сделать вывод о том, что внедрение web-приложений в систему оказания медицинской помощи является актуальным и перспективным направлением развития. Технологии информатики, такие как web-приложения, могут значительно улучшить доступность, эффективность и качество медицинской помощи.

В ходе работы был проведен обзор существующих систем ОНМП в России, а также рассмотрены преимущества использования фреймворка React и библиотеки компонентов MUI для разработки frontend-части приложения. Описаны основные компоненты, такие как поисковая строка, аккордеоны и checkbox, и объяснены их преимущества в контексте улучшения пользовательского опыта.

В результате работы был разработан интерфейс для web-приложение ОНМП, который предоставляет алгоритмы оказания медицинской помощи и включает в себя удобный функционал для взаимодействия врача с картой вызова. Это приложение позволяет оптимизировать процесс оказания помощи, ускорить принятие решений и повысить эффективность работы медицинского персонала.

Однако, несмотря на все достижения, следует отметить, что разработка и внедрение web-приложений в систему оказания медицинской помощи также встречает некоторые вызовы и ограничения. Это включает в себя вопросы безопасности и конфиденциальности данных, необходимость обучения и привлечения персонала к использованию новых технологий, а также адаптацию приложений под различные устройства и браузеры.

Данная выпускная квалификационная работа имеет практическую значимость и может быть использована в дальнейшем развитии систем оказания медицинской помощи с использованием web-приложений. Использование современных технологий информатики, таких как React и MUI, позволяет улучшить качество и доступность медицинской помощи, а также оптимизировать процессы работы медицинского персонала.