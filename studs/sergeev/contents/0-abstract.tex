\abstract % Структурный элемент: РЕФЕРАТ

\keywords{WEB-СЕРВИС, ВРАЧИ, ОНМП, КАРТА ВЫЗОВА, КЛИЕНТСКАЯ ЧАСТЬ, REACT.JS}

Объектом разработки является Web-сервис для врачей из отделения неотложной медицинской помощи, позволяющий в электронной форме заполнять карты вызова.

С целью достижения поставленной задачи был выполнен анализ функциональных требований, связанных с разработкой сервиса, а также проведено исследование методов, используемых для заполнения карт вызова.

Данное решение было реализовано на языке JavaScript c использованием следующих библиотек и инструментов: 
\begin{itemize}
\item React.js является основной библиотекой для разработки клиентской части web-сервиса.
\item Axios предоставляет удобный и гибкий способ взаимодействия с серверной частью.
\item Redux является библиотекой для управления состоянием приложения.
\end{itemize}

Основным результатом работы, полученным в результате разработки, является функциональная и эффективная система, заменяющая ручную бумажную работу медицинских работников. Реализованный веб-сервис позволяет медицинским работникам легко заполнять, хранить и  осуществлять поиск карт вызова.

Полученные результаты разработки предоставляют немаловажное значение для отделения неотложной помощи, так как позволяют существенно улучшить эффективность и точность работы медицинских работников, а также повысить качество обслуживания пациентов.