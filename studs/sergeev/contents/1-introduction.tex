\introduction % Структурный элемент: ВВЕДЕНИЕ

Актуальность данной работы связана с необходимостью оптимизации процесса работы врачей из отделения неотложной медицинской помощи, особенно тех, которые выезжают на вызовы с использованием машин скорой помощи. В настоящее время эти медицинские специалисты заполняют бумажные карты вызова во время осмотра пациента на дому. Однако, такой подход требует ношения с собой больших объемов бумажной документации, что требует больших затрат времени и ресурсов на их заполнение, обработку и хранение.

Переход к электронному заполнению карт вызова имеет ряд значительных преимуществ. Прежде всего, медицинским работникам будет значительно удобнее носить с собой меньшее количество объемной макулатуры, так как электронная форма позволит им быстро и удобно заполнять карты вызова с использованием мобильных устройств, которые они всегда имеют под рукой. Это значительно повысит мобильность и гибкость работы медицинского персонала, позволяя им быстро и эффективно заполнять и обрабатывать данные.

Второе преимущество заключается в возможности хранения электронных карт вызова в интернете. Вместо физического хранения и поиска бумажных документов, медицинские работники смогут сохранять все карты вызова в электронном виде, обеспечивая быстрый и удобный доступ к ним из любого места. Это также позволит легко распечатывать необходимые карты вызова в случае необходимости, например, для передачи другим специалистам или для ведения медицинской документации.

Основной целью данной работы является разработка клиентской части web-сервиса с использованием технологии React.js. Это позволит создать современный и интуитивно понятный пользовательский интерфейс, который облегчит заполнение карт вызова, ускорит процесс ввода данных и повысит эффективность работы медицинских работников.

Был проведен анализ требований к функциональности и пользовательскому интерфейсу web-сервиса, разработана архитектура решения, реализованы необходимые компоненты и проведено тестирование системы. Также была осуществлена интеграция клиентской части с серверной частью системы для обеспечения полной функциональности и взаимодействия с данными.

В рамках данной исследовательской работы можно определить следующие задачи:
\begin{itemize}
    \item Анализ требований к клиентской части web-сервиса: Эта задача включает в себя изучение требований, предъявляемых заказчиком к функциональности и дизайну клиентской части web-сервиса. Необходимо провести тщательный анализ требований, чтобы полноценно понять ожидания и потребности заказчика.
    \item Проектирование клиентской части: Включает разработку детального плана и архитектуры клиентской части web-сервиса. Эта задача включает определение структуры компонентов, взаимодействия между ними, выбор подходящих библиотек и фреймворков для разработки, а также определение способа организации данных и обработки событий.
    \item Разработка функционала авторизации и аутентификации: Данный функционал позволит медицинским работникам входить в систему с использованием уникальных учетных данных. Каждому пользователю будет предоставлен доступ к рабочей среде, где они смогут создавать и использовать собственные шаблоны, а также просматривать и заполнять свои собственные карты вызова.
    \item Создание интерфейса для заполнения и сохранения карт вызова: Разработка пользовательского интерфейса, позволяющего медицинским работникам вводить информацию о состоянии здоровья пациента, тем самым заполняя карту вызова. Данные должны быть легко вводимы и интуитивно понятны, чтобы ускорить процесс заполнения.
    \item Реализация функционала хранения и организации карт вызова в папках: Готовые, Незавершенные, Архив, Шаблоны. Для организации и удобного отображения карт вызова. Это позволит медицинским работникам быстро найти и просмотреть необходимые карты вызова в соответствующих категориях.
    \item Разработка функционала создания и использования шаблонов: Реализация возможности создания шаблонов для быстрого использования частично заполненных карт вызова. Создавая шаблон, пользователь заполняет только некоторые поля карты, чтобы в дальнейшем использовать данный шаблон как основу для карты вызова.
    \item Реализация Функционала поиска, группировки и сортировки карт вызова: Предоставление возможности быстрого поиска нужной информации на основе различных параметров, таких как дата, название карты. Кроме того, группировка и сортировка карт вызова позволят легко ориентироваться в большом объеме данных, что значительно упростит процесс поиска нужной информации.
    \item Тестирование и отладка системы: Проведение тестирования для проверки функциональности и корректности работы системы. Исправление ошибок и устранение неполадок для обеспечения стабильной работы web-сервиса.
\end{itemize}


В данной работе разработка клиентской части web-сервиса осуществляется с использованием React.js - популярной JavaScript-библиотеки для построения пользовательских интерфейсов. React.js предоставляет эффективные инструменты и подходы для создания интерактивных веб-приложений с удобной обработкой данных и переиспользуемыми компонентами. Данная библиотека базируется на концепции компонентного подхода, где пользовательский интерфейс разбивается на независимые компоненты, каждый из которых может иметь свою логику и состояние. Компоненты могут быть переиспользованы, что упрощает разработку и обслуживание приложения.

Дополнительным инструментом к React был выбран Redux. Это библиотека JavaScript, которая предоставляет инструменты для управления состоянием приложения. Она является популярным выбором для разработки сложных и масштабируемых приложений, особенно тех, которые имеют большое количество взаимосвязанных данных.

Redux основывается на концепции однонаправленного потока данных и централизованного хранилища состояния. Вместо того чтобы хранить состояние внутри компонентов, Redux предлагает вынести состояние в отдельное хранилище. Компоненты могут получать доступ к состоянию из хранилища и обновлять его, используя специальные функции.

В целом, совместное использование React и Redux позволяет создавать масштабируемые, предсказуемые и легко поддерживаемые приложения с удобным управлением состоянием \cite{React&Redux}. 

Помимо этого, был выбрана библиотека Axios, которая предоставляет удобный интерфейс для выполнения HTTP-запросов из браузера. Она позволяет взаимодействовать с внешними API, отправлять запросы на сервер и обрабатывать полученные ответы.

Главная цель Axios - упростить и сделать более эффективной работу с HTTP-запросами. Она предоставляет простые методы для отправки различных типов запросов, таких как GET, POST, PUT, DELETE и другие. Кроме того, Axios предоставляет возможности для установки заголовков запроса, обработки ошибок, использования интерсепторов и других функций.

Разработка данного сервиса основывалась на архитектурной методологии Feature-Sliced Design, которая помогает структурировать код и организовать проект на основе функциональных возможностей. Она призвана улучшить масштабируемость, переиспользуемость и поддерживаемость кодовой базы. Основная идея данной методологии заключается в том, что проект разбивается на слои, каждый из которых имеет свою "зону ответственности". Каждый модуль содержит все необходимые компоненты, состояние, стили и логику, связанные с этой функциональностью. Это позволяет разработчикам работать над каждой фичей независимо друг от друга.

Результатом данной работы будет функциональный web-сервис, предоставляющий возможность электронного заполнения и хранения карт вызова для медицинских работников из отделения неотложной помощи. Ожидается, что данное решение значительно улучшит процесс работы медицинских учреждений, сократит временные затраты на бумажную работу и повысит качество медицинской помощи пациентам.

Таким образом, данная дипломная работа имеет практическую значимость и является важным шагом в совершенствовании сферы оказания неотложной медицинской помощи, предлагая инновационное решение, которое отсутствует на рынке.
