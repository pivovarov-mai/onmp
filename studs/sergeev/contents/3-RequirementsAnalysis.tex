\section{АНАЛИЗ ТРЕБОВАНИЙ ЗАКАЗЧИКА К ФУНКЦИОНАЛЬНОСТИ И ДИЗАЙНУ}

\subsection{Описание требований к интерфейсу заполнения карт вызова}

Эффективное и надежное заполнение карт вызова является важным процессом для медицинских работников, поскольку это напрямую влияет на качество медицинской помощи. Интерфейс, который будет использоваться для заполнения карт вызова, должен быть специально разработан, чтобы обеспечить максимальную эффективность и удобство использования.

Для удовлетворения этих требований заказчиком было предложено несколько критериев, которым должен соответствовать медицинский интерфейс:

\begin{itemize}
    \item Интерфейс должен быть легким в освоении и использовании даже для пользователей с минимальными навыками работы с компьютером. Он должен предоставлять понятные и логично расположенные элементы управления, а также ясные инструкции для заполнения каждого поля.
    \item Интерфейс должен быть организован логически, с ясной и понятной структурой разделов и подразделов, отражающих последовательность заполнения карты вызова. Навигация должна быть интуитивной и обеспечивать быстрый переход между разделами, чтобы медицинскому работнику было удобно перемещаться по карте вызова.
    \item Интерфейс должен предоставлять функциональность сохранения заполненной информации, чтобы медицинским работникам не приходилось вводить одни и те же данные снова и снова при каждом новом вызове. Также желательно наличие функции автозаполнения полей на основе предыдущих записей или шаблонов.
    \item Интерфейс должен быть адаптивным и оптимизированным для работы на различных устройствах, включая компьютеры, планшеты и смартфоны. Это обеспечит удобство использования сервиса как в офисной среде, так и на мобильных платформах, когда медицинские работники находятся на вызове.
    \item Интерфейс должен предоставлять возможность интеграции с дополнительными функциональностями, которые помогут медицинским работникам в процессе заполнения карты вызова. Например, он может предоставлять доступ к алгоритмам оказания помощи, таблицам дифференциальной диагностики, калькуляторам дозировок и справочникам, которые помогут в принятии решений и заполнении необходимых полей.
\end{itemize}

\subsection{Рассмотрение требований к организации папок и шаблонов карт вызова}

При организации папок и шаблонов карт вызова, необходимо учитывать множество требований для обеспечения максимальной эффективности и удобства использования.

В первую очередь, необходимо установить четыре папки хранения карт вызова: Готовые, Незавершенные, Архив и Шаблоны.

В папку Готовые попадают карты после того, как пользователь заполнит все необходимые поля на карте вызова и нажмет кнопку "Сохранить". Это позволяет пользователю быстро найти уже готовые карты вызова и не тратить время на поиск незавершенных карт.

В папку Незавершенные попадают карты, когда пользователь возвращается в каталог карт, закрывает вкладку или не нажимает кнопку "Сохранить". При этом, уже заполненные данные сохраняются в процессе заполнения в базу данных, чтобы пользователь мог продолжить заполнение карты вызова с того места, где он остановился. Это позволяет избежать потери уже заполненных данных и повторного заполнения всех полей заново.

В папку Архив попадают карты после того, как они были распечатаны. Это гарантирует сохранность уже готовых карт вызова, которые могут понадобиться в будущем для анализа статистики и других целей.

Папка Шаблоны предоставляет дополнительный функционал создания шаблона с частичным заполнением полей карты вызова. Это позволяет пользователям создавать базовые шаблоны карт вызова, которые можно использовать в будущем как основу для новых карт вызова. Это значительно ускоряет процесс заполнения карт вызова, так как многие поля уже будут заполнены заранее.

Такая организация папок и шаблонов карт вызова обеспечит удобство хранения, поиска и использования карт, а также улучшит процесс работы медицинского персонала, обеспечивая быстрый доступ к необходимым данным и возможность повторного использования шаблонов.

Дополнительные требования к функциональности организации папок и шаблонов карт вызова:

Сортировка по дате:

\begin{itemize}
    \item Возможность сортировки карт вызова внутри каждой папки по дате их создания или последнего изменения.
    \item Медицинским работникам предоставляется возможность выбора порядка сортировки (по возрастанию или убыванию даты).
\end{itemize}

Сортировка по названию карты:

\begin{itemize}
    \item Возможность сортировки карт вызова внутри каждой папки по их названию.
    \item Медицинским работникам предоставляется возможность выбора порядка сортировки (по алфавиту, по возрастанию или убыванию).
\end{itemize}

Группировка:

\begin{itemize}
    \item Возможность группировки карт вызова по определенным параметрам, например, по дате и названию карты.
    \item Медицинским работникам предоставляется возможность выбора группировки и просмотра карт вызова внутри каждой группы.
\end{itemize}

Реализация сортировки и группировки в каждой папке улучшит удобство работы с картами вызова, позволит пользователю быстро находить нужную информацию и эффективно управлять списками карт. Это значительно повысит эффективность работы и улучшит пользовательский опыт.
