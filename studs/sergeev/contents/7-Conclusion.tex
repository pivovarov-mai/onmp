\conclusion

В ходе выполнения работы были достигнуты следующие результаты:

В ходе разработки клиентской части web-сервиса был реализован функционал, позволяющий пользователям заполнять карты вызова с учетом всех необходимых данных и сохранять их в соответствующих папках. Были созданы разделы для удобства навигации внутри карты и промежуточного сохранения заполненных полей. Также был разработан функционал организации папок и использования шаблонов, позволяющий пользователям быстро находить готовые карты вызова, восстанавливать незавершенные и архивировать уже выполненные карты. Внедрение данного функционала значительно упрощает и ускоряет процесс работы медицинского персонала, обеспечивая сохранность и доступность данных.

Оценивая полноту решений поставленных задач, можно сделать вывод о том, что в разработанной клиентской части web-сервиса были учтены основные требования к функциональности и удобству использования. Пользователи получили возможность удобно заполнять и сохранять карты вызова, а также организовывать их хранение с помощью папок и шаблонов. Разделение карт на разные разделы облегчает навигацию и повышает эффективность работы с системой.

В разработанной системе реализован функционал передачи данных модулям "Калькулятор дозировок" и "Таблицы Дифференциальной диагностики" на основе заполненных полей "status localis". Это обеспечивает более точные расчеты и диагностические рекомендации, основываясь на данных, введенных пользователем.

Рекомендации по использованию результатов работы включают в себя обучение персонала по использованию разработанной системы, создание документации с подробным описанием функционала и инструкциями по его использованию, а также обеспечение технической поддержки для пользователей системы. Дальнейшее развитие системы может включать расширение функционала, добавление новых модулей и интеграцию с другими системами здравоохранения.

Оценка технико-экономической эффективности внедрения разработанной системы показывает потенциал снижения временных затрат на заполнение и сохранение карт вызова. Автоматизация процесса заполнения и возможность использования шаблонов значительно повышают эффективность работы медицинского персонала. Кроме того, система обеспечивает удобство хранения и поиск готовых карт вызова, что способствует повышению производительности и снижению риска потери информации.

Сравнивая выполненную работу с лучшими достижениями в области, можно сделать вывод, что разработанная клиентская часть web-сервиса обладает современным и удобным пользовательским интерфейсом, а также предоставляет широкий функционал для заполнения, сохранения и организации карт вызова. Она соответствует современным стандартам разработки клиентских приложений и предоставляет удобные инструменты для работы медицинского персонала.

В целом, выполненная работа представляет собой значимый вклад в область разработки систем здравоохранения и обладает высоким уровнем научно-технического достижения. Разработанный функционал позволяет эффективно заполнять и сохранять карты вызова, повышая производительность и качество оказываемой медицинской помощи. Дальнейшее развитие и совершенствование системы могут вносить еще больший вклад в область здравоохранения и улучшать условия работы медицинских специалистов.