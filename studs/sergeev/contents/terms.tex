\newglossaryentry{id1}{ % Нужны разные id, можно ставить просто последовательно
    name={Веб-сервис},
    description={это программное обеспечение, которое предоставляет функциональность и данные через сеть Интернет. Он позволяет взаимодействовать с удаленными компонентами или приложениями посредством сетевых протоколов, обычно HTTP} 
}

\newglossaryentry{id2}{
    name={Интерфейс},
    description={это точка взаимодействия между двумя или более сущностями, позволяющая им обмениваться информацией или взаимодействовать друг с другом. В контексте разработки программного обеспечения, интерфейс определяет способы взаимодействия пользователя с системой или между различными компонентами системы} 
}

\newglossaryentry{id3}{
    name={Клиентская часть},
    description={это часть программного приложения, которая выполняется на стороне клиента, то есть на компьютере пользователя или на устройстве пользователя, таком как веб-браузер или мобильное устройство. Клиентская часть отвечает за представление данных и взаимодействие с пользователем} 
}

\newglossaryentry{id4}{
    name={Фреймворк},
    description={это набор программных инструментов, библиотек и правил, предназначенных для разработки приложений. Фреймворк предоставляет основу и структуру для создания приложений, определяя общую архитектуру, стандарты и методологии разработки} 
}

\newglossaryentry{id5}{
    name={API},
    description={это набор правил и соглашений, которые определяют, как различные компоненты программного обеспечения должны взаимодействовать друг с другом. API определяет методы и форматы данных, которые могут использоваться для обмена информацией и выполнения определенных операций} 
}

\newglossaryentry{id6}{
    name={Frontend},
    description={это термин, который используется в веб-разработке для обозначения части проекта, которая отвечает за создание и отображение пользовательского интерфейса. Он включает в себя все элементы, которые видит пользователь и с которыми он взаимодействует при использовании веб-приложений или сайтов} 
}

\newglossaryentry{id7}{
    name={JavaScript},
    description={это высокоуровневый, интерпретируемый язык программирования, который широко используется для создания динамических веб-страниц. Он предоставляет возможности для разработки интерактивных элементов на веб-странице и взаимодействия с пользователем} 
}

\newglossaryentry{id8}{
    name={HTTP-запросы},
    description={это средство взаимодействия между клиентскими приложениями и серверами в сети Интернет. Протокол HTTP определяет стандартные методы и форматы запросов, которые клиенты могут отправлять серверам для получения данных или выполнения определенных операций} 
}

\newglossaryentry{id9}{
    name={UX-дизайн},
    description={относится к процессу проектирования и улучшения взаимодействия пользователей с продуктом или сервисом, с целью обеспечения максимально положительного и удовлетворительного опыта пользователя. Он фокусируется на создании продуктов, которые легко и интуитивно понятны, удобны в использовании и соответствуют потребностям и ожиданиям пользователей} 
}
