\introduction % Структурный элемент: ВВЕДЕНИЕ

В современной медицине, врачи скорой помощи часто сталкиваются с ситуациями, когда им необходимо быстро получить доступ к разнообразным справочным материалам. Эти материалы могут значительно облегчить и ускорить процесс оказания помощи пациентам, но в настоящее время медицинскому персоналу приходится использовать большое количество физических справочников или электронных документов. В связи с этим возникают определенные проблемы, такие как затрудненный поиск необходимой информации, ее недоступность или увеличивающееся время, затрачиваемое на обработку информации.

Данный проект является актуальным, поскольку медицинские справочники часто представлены в виде бумажных книг или электронных документов, что затрудняет их использование и доступность для медицинского персонала. Целью проекта является разработка и внедрение справочников в web-приложение для ОНМП. Такая интеграция позволит значительно ускорить поиск информации для медицинского персонала. Они смогут быстро находить необходимую информацию о диагнозах, чтобы назначить лечение.

В рамках проекта будет осуществлена разработка и внедрение справочников в web-приложение с учетом требований по фильтрации и группировке информации. Это позволит обеспечить доступ медицинского персонала к актуальной и надежной медицинской информации в режиме реального времени. Таким образом, проект позволит улучшить процесс принятия решений и повысить эффективность работы медицинского персонала.

Медицинские справочники — это незаменимый инструмент для медицинского персонала. Они предоставляют детальные сведения о симптомах, диагностических методах, лечении и других аспектах заболеваний. Однако, медицинских справочников не всегда удобно использовать в бумажном виде. Большинство справочников не помещается в кармане, носить с собой их тяжело, а их просмотр занимает много времени и не всегда удобен в условиях неотложной медицинской помощи.

Поэтому, интеграция медицинских справочников в web-приложение является логичным шагом в развитии медицинской науки. Это позволит медицинскому персоналу более точно и оперативно оценивать состояние пациента и предлагать соответствующие медицинские решения. Перенос справочных материалов в электронный формат и их включение в специализированное web-приложение значительно упростит процесс поиска информации и улучшит доступность для медицинского персонала. Теперь медицинским работникам, особенно в отделении неотложной медицинской помощи, не нужно будет тратить время на поиск в различных источниках или внутренних базах данных.

Однако, просто перенести справочники в электронный вид недостаточно. Важно учесть особенности медицинской деятельности и предоставить медицинскому персоналу удобный и быстрый доступ к необходимой информации. Использование web-приложения с встроенным справочником позволит медицинскому персоналу быстро получить проверенную и авторитетную информацию, что помогает снизить возможность ошибок в процессе диагностики и лечения пациентов. Внедрение такого решения в web-приложение способствует стандартизации процедур и обеспечению актуальности и достоверности информации. Более того, web-приложение может предоставить дополнительные возможности, такие как доступ к медицинским журналам, обмен мнениями с коллегами или ведение электронной медицинской карты пациента.

В ходе выполнения проекта были решены следующие задачи:
\begin{itemize}
    \item Изучение потребностей и требований медицинского персонала отделения неотложной медицинской помощи в отношении информационных ресурсов;
    \item Анализ существующих медицинских справочников;
    \item Определение функциональности и структуры справочника с учетом требований по фильтрации и группировке информации;
    \item Выбор оптимального стека технологий для разработки и интеграции справочника в web-приложение;
    \item Сбор справочных материалов, доступных в электронной форме, для создания базы данных справочника;
    \item Разработка каркаса web-справочника, обеспечивающего удобный поиск и навигацию по информации;
    \item Интеграция разработанного каркаса в web-приложение отделения неотложной медицинской помощи;
    \item Оценка результатов и сравнение их с начальными требованиями.
\end{itemize}

Для успешной реализации проекта было выбрано разработка web-приложения с использованием фреймворка React.js и библиотеки MUI, содержащей готовые компоненты.

Основными результатами работы являются:
\begin{itemize}
    \item Web-справочник, интегрированный в web-приложение для заполнения медицинских карт отделения неотложной медицинской помощи, который предоставляет материалы для врачей скорой помощи;
    \item Упрощение и ускорение процесса доступа к справочным материалам, что позволяет врачам оперативно принимать решения и оказывать помощь пациентам.
\end{itemize}

Ожидается, что внедрение данного web-приложения значительно повысит эффективность работы врачей скорой помощи и улучшит качество предоставляемой медицинской помощи пациентам. Это произойдет благодаря улучшенной системе управления медицинскими данными и снижению времени, необходимого для обработки пациентов. Кроме того, приложение обеспечит большую точность диагнозов и назначений лечения, что сократит количество ошибок и повысит доверие пациентов к медицинским работникам. В общей сложности, внедрение этого web-приложения принесет большую пользу для пациентов и медицинских работников.
