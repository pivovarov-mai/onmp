\section{ПОСТАНОВКА ЗАДАЧИ}

\subsection{Потребность в улучшении доступности и эффективности медицинской информации}

Современное медицинское обслуживание требует быстрого и удобного доступа к актуальным медицинским справочникам. Однако, в настоящее время, многие справочники представлены в виде бумажных книг или электронных документов, что затрудняет их использование и доступность для медицинского персонала. Для решения этой проблемы необходимо разработать web-справочник, интегрированный с системой отделения неотложной медицинской помощи, который будет упорядочивать и предоставлять все необходимые данные медицинскому персоналу. Такой справочник обеспечит быстрый поиск информации и повысит эффективность работы медицинского персонала.

Однако, следует отметить, что существуют проблемы, которые необходимо решить в процессе разработки и внедрения медицинского справочника в web-приложение отделения неотложной медицинской помощи. Одной из таких проблем является ограниченный доступ к актуальной информации. Традиционные справочники, включая печатные и электронные версии, могут быть устаревшими или не содержать достаточно широкий спектр информации. Это создает трудности при принятии решений в срочных ситуациях и может приводить к задержкам в оказании помощи.

Внедрение медицинского справочника в web-приложение позволит обеспечить медицинскому персоналу быстрый и удобный доступ к актуальным статьям, протоколам, рекомендациям и другой справочной информации. Это улучшит качество медицинской помощи и поможет принимать более обоснованные решения. Важно подчеркнуть, что разработка web-справочника — это не только организация доступа к информации, но и возможность ее постоянного обновления, что позволит всегда иметь актуальные данные и максимально точно оказывать медицинскую помощь.

Другая проблема, которую необходимо решить, связана со временем, затрачиваемым на поиск информации. Врачи скорой помощи часто сталкиваются с необходимостью быстро находить и получать необходимую информацию для оказания срочной помощи. Использование web-приложения с интегрированным справочником значительно сократит время поиска информации и упростит процесс фильтрации и выбора релевантных данных. Благодаря быстрой доступности актуальной информации уменьшаются задержки в предоставлении помощи и улучшаются результаты лечения у пациентов.

Одной из основных сложностей, с которой сталкиваются медицинские работники при поиске информации, является ее поиск и фильтрация по конкретным симптомам, заболеваниям и лекарствам. Различные источники информации могут предоставлять различные данные, что затрудняет их сопоставление и использование. Разработка web-приложения с функцией поиска и фильтрации позволит медицинскому персоналу быстро и точно находить необходимую информацию, основываясь на конкретных параметрах и требованиях. Такой функционал справочника поможет снизить задержки в оказании неотложной медицинской помощи и улучшит точность диагностики и назначения лечения.

Кроме того, недостаточная стандартизация и согласованность информации являются серьезными проблемами в медицине. Различные источники информации могут предоставлять противоречивые и неоднозначные данные, что может привести к путанице и ошибкам в принятии решений. Разработка web-приложения с единым и проверенным источником медицинской информации поможет стандартизировать процедуры и обеспечить согласованность данных, доступных медицинскому персоналу. Это позволит снизить риск возникновения ошибок в диагностике и лечении пациентов.

Важно отметить, что недоступность актуальной и достоверной медицинской информации может повысить риск ошибок и негативных последствий для пациентов. Неправильное применение процедур или назначение неподходящего лечения может иметь серьезные последствия для здоровья и исходов пациентов. Поэтому разработка и внедрение в web-приложения отделения неотложной медицинской помощи медицинского справочника является неотъемлемой частью улучшения качества неотложной медицинской помощи.

Внедрение web-справочника в систему отделения неотложной медицинской помощи поможет оптимизировать и упростить доступ к медицинской информации, что, в свою очередь, позволит значительно повысить качество предоставления медицинской помощи и снизить риск ошибок и задержек в оказании помощи. Кроме того, разработка web-приложения с функцией поиска и фильтрации позволит находить необходимую информацию быстро, точно и на основании конкретных требований. Это является критически важным компонентом для оказания неотложной медицинской помощи и повышения качества медицинской помощи в целом.

\subsection{Анализ существующих подходов к решению проблемы}
В области онлайн-медицинских справочников существует множество ресурсов, которые предлагают информацию о различных заболеваниях, лекарствах, процедурах и других медицинских вопросах. Однако, не все они обладают необходимыми функциональными возможностями и гибкостью настройки, что может затруднять и усложнять поиск и получение нужной информации.

Среди представленных на рынке медицинских справочников можно выделить несколько типов. Один из них - это базовые справочники, которые позволяют осуществлять поиск и просмотр статей, такие как онлайн-ресурсы, медицинские порталы, электронные библиотеки или базы данных. Они обладают широким спектром медицинских материалов, но могут ограничивать пользователей в гибкости настройки и добавления новых материалов.

Другой тип справочников предлагает расширенные функции, такие как фильтрация и сортировка данных, что может быть полезным для быстрого и точного поиска информации. Однако такие решения могут потребовать сложной интеграции и иметь ограниченный доступ к актуальной информации, особенно если они не связаны напрямую с источниками медицинских данных. Кроме того, возможности настройки и добавления новых материалов также могут быть ограничены.

Несмотря на некоторые недостатки, медицинские справочники имеют важные преимущества, такие как доступ к актуальной медицинской информации, удобство использования и возможность получения справочных данных в любое время. Однако, некоторые из них могут иметь недостатки, такие как ограниченный объем информации, отсутствие персонализации, невозможность обновления данных в реальном времени и ограниченный доступ к некоторым функциям без платной подписки.

В рамках проекта мы сосредоточимся на создании интегрированного web-приложения медицинского справочника, который преодолевает эти недостатки. Приложение будет обеспечивать быстрый доступ к информации о заболеваниях и их симптомах, соблюдая актуальные медицинские стандарты, а также предлагать удобный интерфейс и быстрый доступ к данным. Оно также будет обладать гибкостью настройки и возможностью обновления данных, чтобы обеспечить врачам всегда актуальную информацию. Все это позволит сделать приложение максимально полезным и удобным для пользователей.

Web-приложения медицинских справочников могут варьироваться по функционалу и целям. Рассмотрим некоторые из них:
\begin{itemize}
    \item Основные медицинские справочники: этот тип справочника предоставляет общую медицинскую информацию, включая симптомы, диагнозы, лечение и рекомендации. Он может включать информацию о различных заболеваниях, медицинских процедурах, лекарственных препаратах и т.д. Такой справочник позволяет врачам быстро получить доступ к основным медицинским знаниям и руководствам. Недостатком таких справочников является то, что они могут быть слишком общими и не всегда содержать достаточно деталей или актуальной информации для специфических случаев или редких заболеваний. Важно обеспечить актуализацию и достоверность информации, чтобы предотвратить ошибки в диагностике или лечении;
    \item Экстренные справочники: эти справочники ориентированы на оказание первой помощи и управление экстренными ситуациями. Они содержат информацию о действиях, необходимых в случае серьезных травм, сердечного приступа, инсульта и других критических состояниях. Такие справочники могут включать шаг за шагом инструкции, видеоуроки и иллюстрации, чтобы помочь врачам и работникам скорой помощи принять быстрые и правильные решения. Такие справочники ограничены тем, что некоторые ситуации могут требовать индивидуального подхода и консультации с медицинскими специалистами;
    \item Фармакологические справочники: этот тип справочника содержит информацию о лекарственных препаратах, включая их названия, дозировки, побочные эффекты, противопоказания и инструкции по применению. Такие справочники помогают врачам подобрать правильные лекарства для конкретных пациентов и обеспечивают информацию о взаимодействии препаратов и других особенностях. Такие справочники ограничены тем, что принятие решений о лечении требует оценки индивидуальных особенностей пациента и консультации с медицинскими специалистами;
    \item Справочники по специализированным областям: такие справочники посвящены конкретным специализированным областям медицины, таким как педиатрия, гинекология, кардиология, онкология и другие. Они содержат специфическую информацию, связанную с данной областью медицины, и могут быть полезными для специалистов, работающих в этих областях.
\end{itemize}

После консультации с врачом, работающим в отделении неотложной медицинской помощи, было принято решение о создании справочника, включающегося в список общих медицинских справочников. Целью данного справочника является обеспечение быстрого доступа к информации о болезнях и их симптомах, чтобы облегчить и ускорить процесс диагностики в рамках работы скорой помощи, с учетом последних медицинских стандартов. Для достижения этой цели, приложение должно быть удобным в использовании и обеспечивать быстрый доступ к данным, чтобы обеспечить эффективность и оперативность в работе. Важно также обеспечить постоянное обновление и актуализацию информации, чтобы врачи всегда имели доступ к последним рекомендациям и протоколам.

При реализации проекта, интегрированный в web-приложение медицинский справочник получит преимущества перед уже существующими решениями, так как он будет иметь возможность интеграции с текущим приложением для электронного оформления медицинских карт. Это означает, что врачи смогут использовать справочник непосредственно в процессе работы с пациентами, без необходимости переключаться между разными системами. 

\subsection{Обоснование цели и задач, техническое задание на разработку}
В настоящее время врачи скорой помощи работают в условиях, требующих оперативности и своевременных решений. Для того чтобы быстро получить необходимую информацию, им необходим быстрый доступ к актуальной информации, которая должна быть доступна на различных устройствах, включая смартфоны и планшеты. Кроме того, учитывая динамическую природу работы врачей скорой помощи, важно, чтобы справочник был легко доступен и имел интуитивно понятный интерфейс.

Разработка справочника, интегрированного в web-приложение для заполнения медицинских карт, позволит врачам скорой помощи получать быстрый доступ к необходимой информации, что позволит оптимизировать процесс оказания медицинской помощи, ускорить принятие решений и повысить качество обслуживания пациентов. Однако, для того чтобы реализовать эту идею, необходимо учитывать множество требований, таких как фильтрация и группировка информации, удобный интерфейс и т.д.

Поэтому целью данной работы является внедрение медицинских справочников в web-приложение отделения неотложной медицинской помощи с учетом всех вышеперечисленных требований. Такой подход позволит медицинскому персоналу быстро находить необходимую информацию, что повысит их эффективность и улучшит качество обслуживания пациентов.

Для реализации цели работы были поставлены следующие задачи:
\begin{itemize}
    \item Подготовка тестовых данных;
    \item Разработка макета web-страницы;
    \item Разработка каркаса web-справочника в соответствии с макетом с основными функциональными компонентами, такими как поисковая строка, категоризация статей;
    \item Настройка взаимодействия между web-приложением и удаленным web-сервером, обеспечивающее передачу и получение данных.
\end{itemize}

Также web-приложение должно иметь интуитивно понятный интерфейс, который обеспечит удобный поиск, фильтрацию и просмотр информации. Web-приложение должно быть интегрировано в существующую систему электронного оформления медицинских карт, чтобы врачи могли получать доступ к справочной информации без необходимости переключения между различными приложениями.

По завершении проекта, ожидается, что врачи скорой помощи смогут быстро получать доступ к актуальной и проверенной медицинской информации. Более того, интеграция существующего приложения для электронного оформления медицинских карт позволит оптимизировать процесс оказания медицинской помощи, значительно упростить процесс работы и повысить эффективность обслуживания пациентов в отделении неотложной медицинской помощи.