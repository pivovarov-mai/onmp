\conclusion

В рамках выпускной квалификационной работы бакалавра был разработано был разработан web-справочник, который предоставляет пользователям быстрый и надежный доступ к актуальным данным. Система успешно реализована и прошла тестирование, демонстрируя высокую эффективность и удобство использования.

Разработанный web-справочник обеспечивает пользователей достоверной и актуальной информацией, позволяя быстро находить нужные данные. Разделение информации на категории и функция поиска обеспечивают удобство поиска и просмотра информации.  Дополнительно, в разработанном web-справочнике предусмотрены функции добавления, редактирования и удаления статей. Это позволяет администраторам и соответствующему медицинскому персоналу актуализировать содержимое справочника, вносить изменения и обновления, а также удалять устаревшие или неточные записи.

Функция добавления статей позволяет расширять объем и разнообразие информации, включая новые медицинские данные. Это дает возможность справочнику быть гибким и обновляемым инструментом, отражающим последние достижения и новейшие стандарты в медицине. Функция редактирования статей позволяет вносить изменения в уже существующую информацию. Если появляются новые исследования, рекомендации или изменения в статьях, медицинский персонал может обновить статьи, чтобы отразить эти изменения. Таким образом, справочник остается актуальным и достоверным и может быть использован в работе медицинского персонала скорой помощи. Функция удаления статей позволяет удалять записи, которые утратили актуальность или стали неприменимыми. Например, если появляются новые протоколы или рекомендации, которые полностью заменяют предыдущую информацию, устаревшие статьи могут быть удалены, чтобы избежать путаницы и неправильного применения.

Все эти функции добавления, редактирования и удаления статей в справочнике обеспечивают его актуальность, достоверность и гибкость, делая его ценным инструментом для медицинского персонала скорой помощи.

Для дальнейшего развития и совершенствования web-справочника рекомендуется следующее:
\begin{itemize}
    \item расширение базы данных и регулярное обновление информации, чтобы обеспечить актуальность и полноту данных;
    \item введение функции автоматического обновления справочника для обеспечения доступа к последней информации.
\end{itemize}

Разработанный web-справочник является значимым вкладом в область медицинской информатики и обладает высоким потенциалом для улучшения условий работы медицинских специалистов и качества медицинской помощи.