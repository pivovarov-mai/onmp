\section{ПРАКТИЧЕСКАЯ ЧАСТЬ}

\subsection{Методы и инструменты разработки}

В данном разделе работы проводилось детальное исследование возможностей библиотеки PyFPDF для создания различных типов объектов в pdf-документе, включая текст, изображения и таблицы, а также форматирование этих объектов.

Для создания текстовых объектов, PyFPDF предоставляет методы, такие как SetFont() для установки шрифта и его стиля, SetFontSize() для определения размера шрифта, и SetTextColor() для установки цвета текста, способ использования продемонстрирован на рисунке 1.

\begin{figure}
\begin{lstlisting}[language=Python]
pdf = FPDF()
pdf.AddPage()
pdf.SetFont('Arial', 'B', 12)
pdf.SetFontSize(14)
pdf.SetTextColor(0, 0, 0)  # черный цвет текста
pdf.Cell(0, 10, 'Пример текстового объекта', 0, 1, 'C')
\end{lstlisting}
\caption{Методы работы с текстом}
\label{src:src1}
\end{figure} 

Для вставки изображений, можно использовать метод Image() библиотеки PyFPDF, представленный на рисунке 2. Этот метод позволяет указать путь к изображению, его координаты и размеры.

\begin{figure}
\begin{lstlisting}[language=Python]
pdf = FPDF()
pdf.AddPage()
pdf.Image('path/to/image.jpg', x=10, y=10, w=100, h=100)
\end{lstlisting}
\caption{Метод вставки изображений}
\label{src:src2}
\end{figure} 

На рисунке 3 показаны методы  Cell() и MultiCell(). Метод Cell() используется для создания одиночной ячейки таблицы, а MultiCell() - для создания ячейки с поддержкой переноса текста на новую строку. 

\begin{figure}
\begin{lstlisting}[language=Python]
pdf = FPDF()
pdf.AddPage()
pdf.SetFont('Arial', 'B', 12)
pdf.Cell(40, 10, 'Заголовок 1', 1)
pdf.Cell(60, 10, 'Заголовок 2', 1)
pdf.Cell(40, 10, 'Заголовок 3', 1)
pdf.Ln()

data = [['Ячейка 1', 'Ячейка 2', 'Ячейка 3'],
        ['Ячейка 4', 'Ячейка 5', 'Ячейка 6']]

for row in data:
    for item in row:
        pdf.Cell(40, 10, item, 1)
    pdf.Ln()

\end{lstlisting}
\caption{Методы работы с таблицами}
\label{src:src3}
\end{figure} 

На рисунке 4 показано создание разделов и подразделов в документе с помощью методов AddSection() и AddSubSection(). Эти методы позволяют организовать структуру документа и создать иерархию информации.

\begin{figure}
\begin{lstlisting}[language=Python]
pdf = FPDF()
pdf.AddPage()
pdf.AddSection(0, 'Раздел 1')
pdf.Cell(0, 10, 'Содержимое раздела 1')
pdf.AddSubSection(1, 'Подраздел 1.1')
pdf.Cell(0, 10, 'Содержимое подраздела 1.1')
\end{lstlisting}
\caption{Методы структурирования документа}
\label{src:src4}
\end{figure} 

Также, для управления макетом страницы и расположением объектов на ней, библиотека PyFPDF предоставляет методы SetMargins() для установки отступов границы, SetAutoPageBreak() для автоматического переноса курсора на следующую строку и SetXY() позволяющий установить курсор в заданной позиции для вывода объектов. Пример использования упомянутых методов приведен на рисунке 5.

\begin{figure}
\begin{lstlisting}[language=Python]
pdf = FPDF()
pdf.AddPage()
pdf.SetMargins(20, 20, 20)  # установка отступов страницы в 20 мм со всех сторон
pdf.SetAutoPageBreak(auto=True, margin=15)  # автоматический перенос на новую страницу
pdf.SetXY(30, 30)  # установка текущей позиции курсора на 30;30
\end{lstlisting}
\caption{Методы структурирования документа}
\label{src:src5}
\end{figure} 

В процессе обзора было продемонстрировано, что библиотека PyFPDF предоставляет мощные и гибкие инструменты для создания различных типов объектов в pdf-документе, а также различные методы для форматирования и стилизации объектов, что позволяют легко настраивать и изменять внешний вид и расположение элементов, что особенно важно при создании динамически изменяемого документа.

Одним из преимуществ библиотеки PyFPDF является ее простота в использовании. Структура кода понятна и интуитивно понятна даже новым пользователям. Множество примеров и документация, предоставленные разработчиками, облегчают изучение и позволяют быстро приступить к созданию pdf-документов.

Кроме того, библиотека PyFPDF поддерживает множество функций, таких как создание разделов и подразделов, добавление гиперссылок, управление макетом страницы и другие. Эти дополнительные возможности позволяют создавать более сложные и информативные документы с учетом специфических требований медицинской карты.

Таким образом, обзор методов библиотеки PyFPDF подтверждает, что она является мощным инструментом для создания и форматирования pdf-документов, а ее простота в использовании и богатый набор функций делают ее идеальным выбором для решения задачи генерации динамически изменяемой медицинской карты.

\subsection{Алгоритм работы}
Описанная программа принимает список аргументов, представляющих данные медицинской карты, предоставленные сотрудником ОНМП. Ее цель - создать шаблон медицинского заключения и одновременно заполнить его заданными данными. Кроме того, программа следит за тем, чтобы текст помещался в отведенные поля, при необходимости расширяя их и корректируя положение последующих элементов в отчете.

Первым шагом является анализ исходных данных и распределение информации в необходимые поля для медицинского отчета. Программа оценивает размер данных и соответствующим образом корректирует макет шаблона, гарантируя, что вся информация может быть размещена в заранее определенных разделах. Такая динамическая корректировка размеров полей помогает предотвратить переполнение или обрезание текста в итоговом отчете.

После подготовки шаблона программа заполняет его данными, переданными из веб-формы. Она аккуратно вставляет соответствующую информацию, такую как данные пациента, диагноз, процедуры лечения и рекомендации, в соответствующие разделы отчета. Программа также выполняет проверку, чтобы убедиться, что данные вставлены точно и не пропущена важная информация.

После успешного заполнения шаблона программа генерирует окончательный медицинский отчет в формате PDF. Наконец, программа сохраняет созданный PDF-отчет, который сервером отправляется клиенту, запросившему медицинскую карту. Этот шаг гарантирует, что медицинский отчет будет надежно доставлен и доступен пользователю для дальнейшего анализа, архивирования или обмена с другими медицинскими работниками.

Объединяя возможности создания шаблона, заполнения его данными, динамической настройки размеров полей и создания PDF-документа, программа предлагает комплексное решение для эффективного создания медицинских отчетов на основе данных медицинской карты. Она упрощает процесс, минимизирует ручные усилия и повышает точность и согласованность создаваемых отчетов.


